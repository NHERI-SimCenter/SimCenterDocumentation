\label{sec:TInF-installation}

\section{Installing the Turbulent Inflow Tool}

Download the installation package for your operation system from (a single line)
\begin{verbatim}
https://www.designsafe-ci.org/data/browser/public/designsafe.storage.community/
            /SimCenter/Software/TurbulantInflowTool
\end{verbatim}
 SimCenter is providing packages for Windows~8/10 (64 bit version only) and MacOS.  
The installer will place the executable on your system.  On Windows systems, a shortcut will be added to your start menu.
On MacOS, the application is placed in your Applications folder.
\bigskip

For Linux systems, you will need to clone the source from 
\begin{verbatim}
https://github.com/NHERI-SimCenter/TurbulentInflowTool
\end{verbatim}
and compile it yourself performing the following steps:
\begin{quote}
\begin{verbatim}
$ git clone https://github.com/NHERI-SimCenter/TurbulentInflowTool
$ git clone https://github.com/NHERI-SimCenter/SimCenterCommon
$ cd TurbulentInflowTool
$ qmake TurbulentInflowTool.pro
$ make
$ sudo make install
\end{verbatim}
\end{quote}

\section{Other requirements}

JAY: We need a reference to your code on github and basic instructions on how to get OpenFOAM to use them.
