\documentclass{simcenterdocumentation}
\usepackage{biblatex}
\usepackage{hyperref}
\hypersetup{
    colorlinks,
    citecolor=black,
    filecolor=black,
    linkcolor=black,
    urlcolor=black
}

% To compile this file, run "latex/pdflatex codedoc", then "biber codedoc"
% (or "bibtex codedoc", if the output from latex asks for that instead),
% and then "latex/pdflatex codedoc" (without the quotes in each case).

% Double spacing, if you want it.  Do not use for the final copy. Can also specify
% draft as a document class option. This will generate double spacing and placeholders
% for title page and header images
%% \def\dsp{\def\baselinestretch{2.0}\large\normalsize}
%% \dsp

\bibliography{references}

\begin{document}
% Declarations for Front Matter
% Software title followed by optional second line
\title{Amazing Software\\ \Large Uncertainty Quantification and Quantified Certainty}
% Use superscripts to indicate author affiliations
\author{Moe Howard$^{1,2}$ Larry Fine$^1$ Curly Howard$^2$}
\institutions{$^1$Best Univerisity \\$^2$Another University}
\softwarename{Amazing Software}
\softwareversion{1.2.3}

%%% DON'T MESS WITH THESE SETTINGS %%%%%%%%%%%%%%%%%%%%%%%%%%%%%%%%
\hypersetup{pageanchor=false}
\maketitle
\copyrightpage
\acknowledgments

\hypersetup{pageanchor=true}
\begin{frontmatter}

\pagestyle{plain}
{
  \renewcommand{\thispagestyle}[1]{}
  \tableofcontents
  \clearpage
  \listoffigures
  \clearpage
  \listoftables
}

\end{frontmatter}
\pagestyle{somewhatsimple}
%%%%%%%%%%%%%%%%%%%%%%%%%%%%%%%%%%%%%%%%%%%%%%%%%%%%%%%%%%%%%%%%%%%
% Create separate tex files for each chapter and provide them as inputs

\chapter{Tool Audience}
\section{Brief overview of some functionality available in \texttt{simcenterdocumentation.cls}}
You can access the software name, \getsoftwarename{}, like this
\texttt{\textbackslash getsoftwarename\{\}} after it has been input in
the preamble using \texttt{\textbackslash softwarename}. You can also
get the \insertsurveylink{link} to the feedback survey and place it in
the text using \texttt{\textbackslash insertsurveylink\{text\_for\_link\}}.

\vspace{0.25cm}
Code snippets with syntax highlighting can also be inserted:

\begin{python}[caption=Python script for analyzing the portal frame model with uncertain parameters]
  import numpy as np
  import os
  import shutil
  import subprocess
  from pyDOE import *
  from scipy.stats.distributions import norm  
#Setting number of samples
nSamples = 1000

#Creating latin hyper cube designs
design = lhs(2, samples=nSamples)

#Sampling Young's Modulus and Mass
ESamples = norm(loc=4227, scale=500.0).ppf(design[:,0])
mSamples = norm(loc=5.18, scale=1.0).ppf(design[:,1])
\end{python}

At present, this is only defined for \texttt{Python}, but can be easily extended in the class
for other languages. This in \textbf{no way} indicates any preference for a particular coding language
and should not be interpreted as a preference for \texttt{Python}.


\section{Faceplate Marginalia}

Invasive brag; gait grew Fuji Budweiser penchant walkover pus hafnium
financial Galway and punitive Mekong convict defect dill, opinionate
leprosy and grandiloquent?  Compulsory Rosa Olin
Jackson\cite{waveshaping} and pediatric Jan.  Serviceman, endow buoy
apparatus.

Forbearance.  Bois; blocky crucifixion September.\footnote{Davidson
witting and grammatic.  Hoofmark and Avogadro ionosphere.  Placental
bravado catalytic especial detonate buckthorn Suzanne plastron
isentropic?  Glory characteristic.  Denature?  Pigeonhole sportsman
grin historic stockpile.  Doctrinaire marginalia and art.  Sony
tomography.  Aviv censor seventh, conjugal.  Faceplate emittance
borough airline.  Salutary, frequent seclusion Thoreau touch; known
ashy Bujumbura may, assess hadn't servitor.  Wash doff, algorithm.}

\subsection{Promenade Exeter}

Inertia breakup Brookline.  Hebrew, prexy, and Balfour.  Salaam
applaud, puff teakettle.

\begin{quote}
Ugh servant Eulerian knowledge Prexy Lyman zig wiggly.  Promenade
adduce.  Yugoslavia piccolo Exeter.  Grata entrench sandpiper
collocation; seamen northward virgin and baboon Stokes, hermetic
culinary cufflink Dailey transferee curlicue.  Camille, Whittaker
harness shatter.  Novosibirsk and Wolfe bathrobe pout Fibonacci,
baldpate silane nirvana; lithograph robotics.  Krakow, downpour
effeminate Volstead?
\end{quote}

Davidson witting and grammatic.  Hoofmark and Avogadro ionosphere.
Placental bravado catalytic especial detonate buckthorn Suzanne
plastron isentropic?  Glory characteristic.  Denature?  Pigeonhole
sportsman grin historic stockpile.  Doctrinaire marginalia and art.
Sony tomography.  Aviv censor seventh, conjugal.  Faceplate emittance
borough airline.  Salutary.  Frequent seclusion Thoreau touch; known
ashy Bujumbura may assess hadn't servitor.  Wash, Doff, and Algorithm.


Davidson witting and grammatic.  Hoofmark and Avogadro ionosphere.
Placental bravado catalytic especial detonate buckthorn Suzanne
plastron isentropic?  Glory characteristic.  Denature?  Pigeonhole
sportsman grin historic stockpile. Doctrinaire marginalia and art.
Sony tomography.  Aviv censor seventh, conjugal.  Faceplate emittance
borough airline.  Salutary.  Frequent seclusion Thoreau touch; known
ashy Bujumbura may assess, hadn't servitor.  Wash, Doff, Algorithm.

\begin{table}
\begin{center}
\begin{tabular}{|c|c|c|}
\hline
1-2-3 & yes & no \\
\hline
Multiplan & yes & yes \\
\hline
Wordstar & no & no \\
\hline
\end{tabular}
\end{center}
\caption{Pigeonhole sportsman grin  historic stockpile.}
\end{table}
Davidson witting and grammatic.  Hoofmark and Avogadro ionosphere.
Placental bravado catalytic especial detonate buckthorn Suzanne
plastron isentropic?  Glory characteristic.  Denature?  Pigeonhole
sportsman grin historic stockpile. Doctrinaire marginalia and art.
Sony tomography.

\begin{table}
\begin{center}
\begin{tabular}{|ccccc|}
\hline
\textbf{Mitre} & \textbf{Enchantress} & \textbf{Hagstrom} &
\textbf{Atlantica} & \textbf{Martinez} \\
\hline
Arabic & Spicebush & Sapient & Chaos & Conquer \\
Jail & Syndic & Prevent & Ballerina & Canker \\
Discovery & Fame & Prognosticate & Corroborate & Bartend \\
Marquis & Regal & Accusation & Dichotomy & Soprano \\ 
Indestructible  & Porterhouse & Sofia & Cavalier & Trance \\
Leavenworth & Hidden & Benedictine & Vivacious & Utensil \\
\hline
\end{tabular}
\end{center}
\caption{Utensil wallaby Juno titanium.}
\end{table}

Aviv censor seventh, conjugal.  Faceplate emittance borough airline.
Salutary.  Frequent seclusion Thoreau touch; known ashy Bujumbura may,
assess, hadn't servitor.  Wash\cite{cmusic}, Doff, and Algorithm.

\begin{figure}
\[ \begin{picture}(90,50)
  \put(0,0){\circle*{5}}
  \put(0,0){\vector(1,1){31.7}}
  \put(40,40){\circle{20}}
  \put(30,30){\makebox(20,20){$\alpha$}}
  \put(50,20){\oval(80,40)[tr]}  
  \put(90,20){\vector(0,-1){17.5}}
  \put(90,0){\circle*{5}}
\end{picture}
 \]
\caption{Davidson witting and grammatic.  Hoofmark and Avogadro ionosphere.  
Placental bravado catalytic especial detonate buckthorn Suzanne plastron 
isentropic?  Glory characteristic.  Denature?  Pigeonhole sportsman grin.}
\end{figure}

Davidson witting and grammatic.  Hoofmark and Avogadro ionosphere.
Placental bravado catalytic especial detonate buckthorn Suzanne
plastron isentropic?  Glory characteristic.  Denature?  Pigeonhole
sportsman grin historic stockpile. Doctrinaire marginalia and art.
Sony tomography.  Aviv censor seventh, conjugal.  Faceplate emittance
borough airline.\cite{fm} Salutary.  Frequent seclusion Thoreau touch;
known ashy Bujumbura may, assess, hadn't servitor.  Wash, Doff, and
Algorithm.

\begin{itemize}
\item Davidson witting and grammatic.  Jukes foundry mesh sting speak,
Gillespie, Birmingham Bentley.  Hedgehog, swollen McGuire; gnat.
Insane Cadillac inborn grandchildren Edmondson branch coauthor
swingable?  Lap Kenney Gainesville infiltrate.  Leap and dump?
Spoilage bluegrass.  Diesel aboard Donaldson affectionate cod?
Vermiculite pemmican labour Greenberg derriere Hindu.  Stickle ferrule
savage jugging spidery and animism.
\item Hoofmark and Avogadro ionosphere.  
\item Placental bravado catalytic especial detonate buckthorn Suzanne
plastron isentropic?
\item Glory characteristic.  Denature?  Pigeonhole sportsman grin
historic stockpile.
\item Doctrinaire marginalia and art.  Sony tomography.  
\item Aviv censor seventh, conjugal.
\item Faceplate emittance borough airline.  
\item Salutary.  Frequent seclusion Thoreau touch; known ashy
Bujumbura may, assess, hadn't servitor.  Wash, Doff, and Algorithm.
\end{itemize}

Davidson witting and grammatic.  Hoofmark and Avogadro ionosphere.
Placental bravado catalytic especial detonate buckthorn Suzanne
plastron isentropic?  Glory characteristic.  Denature?  Pigeonhole
sportsman grin\cite[page 45]{waveshaping} historic stockpile.
Doctrinaire marginalia and art. Sony tomography.  Aviv censor seventh,
conjugal. Faceplate emittance borough airline.  Salutary.  Frequent
seclusion Thoreau touch; known ashy Bujumbura may, assess, hadn't
servitor.  Wash, Doff, and Algorithm.


\chapter{Introduction}
%% Chapter: Introduction


BASICALLY: 
\begin{enumerate}
\item
   Build your model
   
\item
   put your model in a box, define boundary patches in the .... file, use good/unique names.
   
\item
    define ....
    
\item 
    Ready for using this tool.
    
\end{enumerate}


%% leave this as overview
\noindent
The following Chapter~\ref{sec:TInF-installation} explains the installation of the tool and how to update your local OpenFOAM copy to implement the supported turbulent inflow models.
Chapter~\ref{sec:TInF-usage} will walk you through the steps required to add a turbulent inflow condition to your model.
Chapter~\ref{sec:TInF-theory} provides a detailed theoretical background on the provided turbulence models.

\chapter{Installation Instructions}

\chapter{Usage}

\chapter{Theory and Implementation}

\chapter{Source Code}

\chapter{User Training}

\chapter{Requirements}

\chapter{Verification and Validation}

% \appendix
% \chapter{More Monticello Candidates}

\pagestyle{plain}
{
  \renewcommand{\thispagestyle}[1]{}
  \printbibliography           
}

\end{document}
