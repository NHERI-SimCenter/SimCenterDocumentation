The following section (will) contain examples that verify the functionality of the tool.

Examples
This section provides examples of using EE-UQ for uncertainty quantification of structural analysis models used in earthquake engineering. 
Results of each model is verified against results obtained using other tools.
Two-Dimensional Portal Frame subjected to Gravity and Earthquake Loading
In this example, a simple 2D portal frame model is used to verify the results of $EE-UQ$. 
The model is a linear elastic single-bay single-story model of a reinforced concrete portal frame (\autoref{fig:figure20}). 
The analysis of this model considers both gravity loading and lateral earthquake loading due to El Centro earthquake 
(Borrego Mountain 04/09/68 0230, El Centro ARRAY \#9, 270). 
The original model and ground motion used in this example were obtained from 
\href{http://opensees.berkeley.edu/wiki/index.php/OpenSees_Example_1b._Elastic_Portal_Frame}{example 1b} in OpenSees website, 
and were modified to scale the ground motion record from gravity units (g) to the model units (in/sec2). 
Files for this example are included with the release of the software and are available in the Examples folder in a subfolder called PortalFrame2D.


\begin{figure}[!htbp]
  \centering {
    \includegraphics[width=0.8\textwidth]
    {figs/Figure20.png} }
  \caption{Two-dimensional portal frame model subjected to gravity and earthquake loading}
  \label{fig:figure20}
\end{figure}

To introduce uncertainty in the model, both mass and young’s modulus are assumed to be normally distributed random variables with means and 
standard deviation values shown in Table 1. In this example, the model will be sampled with the Latin Hypercube sampling method using both 
EE-UQ and a Python script (PortalFrameSampling.py) and response statistics from both analyses will be compared.


\begin{table}[hbt!]                       
  \centering
\begin{adjustbox}{max width=\textwidth}            
  \begin{tabular}{lllll}                    
    \toprule          
      Uncertain Parameter & 	Distribution	 &  Mean  &  Standard Deviation \\ \hline
	Nodal Mass, m [kip]	 & Normal & 	5.18	 & 1.0 \\ \hline
	Young’s Modulus, E [ksi] & 	Normal	 & 4227	 & 500.0 \\ \hline
  \end{tabular}
\end{adjustbox}
  \caption{ Uncertain parameters defined in the portal frame model}             
  \label{tab:uncertainty}                 
\end{table}



Modeling uncertainty using EE-UQ can be done using the following steps:
\begin{enumerate}
\item	 Start EE-UQ, click on the simulation tab (SIM) in the left bar to open a building simulation model. Click on choose button in the input script row:

\begin{figure}[!htbp]
  \centering {
    \includegraphics[width=0.8\textwidth]
    {figs/Figure21.png} }
  \caption{Choose building model}
  \label{fig:figure21}
\end{figure}

\item	 Choose the model file Portal2D-UQ.tcl from PortalFrame2D example folder.
\begin{figure}[!htbp]
  \centering {
    \includegraphics[width=0.8\textwidth]
    {figs/Figure22.png} }
  \caption{Choose tcl file}
  \label{fig:figure22}
\end{figure}


\item	 In the list of Clines Nodes edit box, enter “1, 3”. This indicates to EE-UQ that nodes 1 and 3 are the nodes used to obtain EDP at different floor levels (i.e. base and first floor).
\begin{figure}[!htbp]
  \centering {
    \includegraphics[width=0.8\textwidth]
    {figs/Figure23.png} }
  \caption{Select nodes}
  \label{fig:figure23}
\end{figure}

\item Click on the event tab (EVT) in the left bar to open the earthquake event specification tab, select Multiple Existing for loading Type. Click on the add button to add an earthquake event. 
Then click on the choose button to select the event file.
\begin{figure}[!htbp]
  \centering {
    \includegraphics[width=0.8\textwidth]
    {figs/Figure24.png} }
  \caption{Work on EVT tab}
  \label{fig:figure24}
\end{figure}

\item Choose the event file (BM68elc.json) for El Centro earthquake provided in the portal frame 2D example folder.
\begin{figure}[!htbp]
  \centering {
    \includegraphics[width=0.8\textwidth]
    {figs/Figure25.png} }
  \caption{Choose event file}
  \label{fig:figure25}
\end{figure}

\item Now select the random variables tab (RVs) from the left bar, change the random variables types to normal 
and set the mean and standard deviation values of the floor mass and Young’s modulus. 
Notice that EE-UQ has automatically detected parameters defined in the OpenSees tcl file using the pset command and defined them as random variables.
\begin{figure}[!htbp]
  \centering {
    \includegraphics[width=0.8\textwidth]
    {figs/Figure26.png} }
  \caption{Work on RVs tab}
  \label{fig:figure26}
\end{figure}

\item Now click on run, set the analysis parameters, working directory and applications directory and click submit to run the analysis. 
If everything ran successfully the program will automatically open the results tab showing the summary of results (\autoref{fig:figure27}).
\begin{figure}[!htbp]
  \centering {
    \includegraphics[width=0.8\textwidth]
    {figs/Figure27.png} }
  \caption{Run}
  \label{fig:figure27}
\end{figure}

\end{enumerate}



Verification script
A verification script (Listing 1) for propagating the uncertainty was developed in Python and is included in the example folder. 
The script creates 1000 samples for both the Young’s modulus and mass values using Latin Hypercube sampling, 
then modifies the OpenSees model, runs it and stores the output. 
After all the model samples are processed, the script will compute and output the mean and standard deviation values of the peak floor acceleration and peak drift.

{\tiny
\begin{lstlisting}[caption=Python script for analyzing the portal frame model with uncertain parameters]
import numpy as np
import os
import shutil
import subprocess
from pyDOE import *
from scipy.stats.distributions import norm

#Setting number of samples
nSamples = 1000

#Creating latin hyper cube designs
design = lhs(2, samples=nSamples)

#Sampling Young's Modulus and Mass
ESamples = norm(loc=4227, scale=500.0).ppf(design[:,0])
mSamples = norm(loc=5.18, scale=1.0).ppf(design[:,1])

#Initializing output arrays
PFA = []
PID = []
#Reading OpenSees Model
with open ("Ex1b.Portal2D.EQ.tcl", "r") as portalFrameFile:
    portalFrameModel = portalFrameFile.read()

    #Looping through the samples and creating modified models
    for i in range(nSamples):
        sampleName = str(i+1)
        if(os.path.exists(sampleName) and os.path.isdir(sampleName)):
            shutil.rmtree(sampleName)

        os.mkdir(sampleName)
        shutil.copy('BM68elc.acc', sampleName)

        #Modifying the model using sample E and m values
        with open (sampleName + '/Ex1b.Portal2D.EQ.tcl' , "w+") as modifiedFile:
            modifiedModel = portalFrameModel.replace('pset floorMass 5.18', 'pset floorMass ' + str(mSamples[i]))
            modifiedModel = modifiedModel.replace('pset E 4227', 'pset E ' + str(ESamples[i]))
            modifiedFile.write(modifiedModel)

        #Running OpenSees
        subprocess.Popen("OpenSees Ex1b.Portal2D.EQ.tcl", shell=True, cwd=sampleName).wait()

        #Reading Peak Floor Acceleration
        with open (sampleName + '/PFA.out' , "r") as pfaFile:
            PFA.append(float(pfaFile.readlines()[2]))

        #Reading Peak Floor Acceleration
        with open (sampleName + '/PID.out' , "r") as pidFile:
            PID.append(float(pidFile.readlines()[2]))

        #Cleaning up
        shutil.rmtree(sampleName)

#Printing results
print 'Mean Peak Floor Acceleration: ', np.mean(PFA)
print 'Peak Floor Acceleration Std. Dev: ', np.std(PFA)

print 'Mean Peak Drift: ', np.mean(PID)
print 'Peak Drift Std. Dev.: ', np.std(PID)

\end{lstlisting}
}


Verification of results
In this section, the results produced for the portal frame by EE-UQ are verified against the results of running the same problem using the Python script. 
Running the uncertainty quantification problem on the local computer produces the results shown in \autoref{fig:figure28} 
Running the analysis using the sampling Python script produces the results shown in \autoref{fig:figure29}. 
Both results (Mean and standard deviation values of EDPs) are compared in Table 2 and are shown to be in good agreement.
\begin{figure}[!htbp]
  \centering {
    \includegraphics[width=0.8\textwidth]
    {figs/Figure28.png} }
  \caption{Outputs from EE-UQ}
  \label{fig:figure28}
\end{figure}


\begin{figure}[!htbp]
  \centering {
    \includegraphics[width=0.8\textwidth]
    {figs/Figure29.png} }
  \caption{Outputs from PortalFrameSamplying.py script}
  \label{fig:figure29}
\end{figure}




\begin{table}[hbt!]                 
  \centering
\begin{adjustbox}{max width=\textwidth}            
  \begin{tabular}{lllll}                    
    \toprule          
      Engineering Demand Parameter &	 & EE-UQ	& Python Script	 & Percent Difference [\%]  \\ \hline
    
	\multirow{2}{*}{Peak Floor Acceleration [in/$s^2$]} 
	 & Mean &	67.4377	& 67.5448	& 0.16 \\
      & Std. Dev.	& 12.6487	 & 12.5487	& 0.8 \\ \hline
      
      \multirow{2}{*}{Peak Story Drift [x10-3 in]} 
      & Mean &	1.3428 &	1.347 &	0.3 \\
      & Std. Dev.	& 0.2832 &	0.2955	& 4.1	 \\

      \bottomrule      
                            
  \end{tabular}
\end{adjustbox}
  \caption{Features (M=Mandatory, D=Desirable, O=Optional, P=Possible Future)}             
  \label{tab:edp}                 
\end{table}