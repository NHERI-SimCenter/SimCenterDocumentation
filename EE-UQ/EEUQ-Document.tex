\documentclass{simcenterdocumentation}
\usepackage[backend=biber]{biblatex}
\usepackage{hyperref}
\hypersetup{
    colorlinks,
    citecolor=black,
    filecolor=black,
    linkcolor=black,
    urlcolor=blue
}

\usepackage{subfig}
\usepackage{listings,xcolor}
\usepackage{multirow}
\usepackage{adjustbox,lipsum}% to adjust table width

%% SETTING ENVIRONMENT FOR PYTHON CODE SNIPPETS %%%%%%%%%%%%%%%%%%%%%%%%%%%%%%% 
\usepackage[utf8]{inputenc}
% Default fixed font does not support bold face
\DeclareFixedFont{\ttb}{T1}{txtt}{bx}{n}{12} % for bold
\DeclareFixedFont{\ttm}{T1}{txtt}{m}{n}{12}  % for normal

% Custom colors
\usepackage{color}
\definecolor{deepblue}{rgb}{0,0,0.5}
\definecolor{deepred}{rgb}{0.6,0,0}
\definecolor{deepgreen}{rgb}{0,0.5,0}
\usepackage{listings}

% Python style for highlighting
\newcommand\pythonstyle{\lstset{
language=Python,
basicstyle=\ttm,
otherkeywords={self},             % Add keywords here
keywordstyle=\ttb\color{deepblue},
emph={MyClass,__init__},          % Custom highlighting
emphstyle=\ttb\color{deepred},    % Custom highlighting style
stringstyle=\color{deepgreen},
frame=tb,                         % Any extra options here
showstringspaces=false,            %
breaklines=true
}}

% Python environment
\lstnewenvironment{python}[1][]
{
\pythonstyle
\lstset{#1}
}
{}
%%%%%%%%%%%%%%%%%%%%%%%%%%%%%%%%%%%%%%%%%%%%%%%%%%%%%%%%%%%%%%%%%%%%%%%%%%%%%%% 

% To compile this file, run "latex/pdflatex codedoc", then "biber codedoc"
% (or "bibtex codedoc", if the output from latex asks for that instead),
% and then "latex/pdflatex codedoc" (without the quotes in each case).

% Double spacing, if you want it.  Do not use for the final copy. Can also specify
% draft as a document class option. This will generate double spacing and placeholders
% for title page and header images
%% \def\dsp{\def\baselinestretch{2.0}\large\normalsize}
%% \dsp

\bibliography{references}

\begin{document}
% Declarations for Front Matter
% Software title followed by optional second line
\title{EE-UQ\\ \Large The Earthquake Engineering with Uncertainty Quantification (EE-UQ) Application}
% Use superscripts to indicate author affiliations
\author{Frank McKenna, Wael Elhaddad, Chaofeng Wang, Michael Gardner, and Adam Zsarn\'oczay}
%\author{Moe Howard$^{1,2}$ Larry Fine$^1$ Curly Howard$^2$}
\institutions{NHERI SimCenter, UC Berkeley}
\softwarename{EE-UQ}
\softwareversion{2.0.0}

%%% DON'T MESS WITH THESE SETTINGS %%%%%%%%%%%%%%%%%%%%%%%%%%%%%%%%
\hypersetup{pageanchor=false}
\maketitle
\copyrightpage
\acknowledgments

\hypersetup{pageanchor=true}
\begin{frontmatter}

\pagestyle{plain}
{
  \renewcommand{\thispagestyle}[1]{}
  \tableofcontents
  \clearpage
  \listoffigures
  \clearpage
  \listoftables
}

\end{frontmatter}
\pagestyle{somewhatsimple}
%%%%%%%%%%%%%%%%%%%%%%%%%%%%%%%%%%%%%%%%%%%%%%%%%%%%%%%%%%%%%%%%%%%
% Create separate tex files for each chapter and provide them as inputs


\chapter{About}
\label{chap:about}
We live in a world in which information, data, is disseminated almost exclusively via the world wide web (WWW).  For researchers/practitioners in natural hazards engineering (NHE) that data is diverse, dynamic, distributed and often quite large. While manual gathering and processing of limited data sets is possible, the proliferation of data has overwhelmed the field of NHE greatly limiting its use. The SimCenter is releasing a series of applications that will enable researchers in field of natural hazards engineering (NHE) to incorporate these online resources in their numerical simulations through the use of scientific workflows: “scientific workflows are used to automate the analysis of data through multiple, distributed data resources in order to execute complex in silico experiments”.  [“Scientific workflows” paper by Katy Wolstencroft, Paul Fisher, David De Roure and Carole Goble.] The SimCenter is providing the NHE community with limited scientific workflow systems. Scientific Workflow Systems these are applications to build, launch, and monitor  scientific workflows. If one considers a scientific workflow as a sequence of applications being invoked akin to a jigsaw puzzle, as abstractly shown in \Cref{fig:jigsaw}, by providing applications to build. launch, and monitor scientific workflow, we mean an application that allows users to select from different applications for each jigsaw, then launch and monitor an application  which will run each application in the workflow passing the needed input and output data between the applications. The SimCenter systems are limited in that, unlike existing systems, limit the number and how the applications are put together.  

\begin{figure}[!htbp]
  \centering {
    \includegraphics[width=0.75\textwidth]
    {images/jigsaw.png} }
  \caption{Abstraction of Scientific Workflow Application}
  \label{fig:jigsaw}
\end{figure}
 
Numerical simulations of real world phenonomena involve uncertainties and as a consequence need to generate measures of the uncertainty in the computed response. Those in NHE utilize a number of different uncertainty quantification (UW) methods, typically categorized into one of the following: 1) Forward methods, 2) Sensitivity methods,  and 3) Reliability methods.  All require that deterministic numerical simulations run repeatedly with different inputs. In SimCenter applications UQ engines, as shown in \Cref{fig:uqEngine},  run these simulations. The simulations can be run on either the user’s desktop computer or on high performance computers (HPC) made available to NHE community  through DesignSafe-ci.

 \begin{figure}[!htbp]
  \centering {
    \includegraphics[width=0.75\textwidth]
    {images/uqEngine.png} }
  \caption{UQ Engine}
  \label{fig:uqEngine}
\end{figure}


The SimCenter is providing a number of scientific workflow  systems:  quoFEM, EE-UQ, WE-UQ, PBE, and RDT. Due to the limited programming resources available, these applications all are built using a common software framework. The framework, as shown in \Cref{fig:SimCenterSoftware}, comprises elements for Cloud Computing (limited to sending data and interfacing with remote service providers), UQ, SAM (structural analysis modeling), EVENT (earthquake, wind, tsunami, and storm surge), EDP (engineering demand parameters), FEM (finite element analysis and coupled Computational Fluid Dynamics-Finite Element Analysis), DL (damage and loss prediction). The framework also includes databases: DL( fragility curves for damage and loss calculations), Experimental and Simulation Models, and BE (built environment). In order to ingest data into the database BE is provided by software SimCEnter is developing as part of BRAILS.


\begin{figure}[!htbp]
  \centering {
    \includegraphics[width=0.75\textwidth]
    {images/SimCenterSoftware.png} }
  \caption{SimCenter Software Applications and Components}
  \label{fig:SimCenterSoftware}
\end{figure}

 This document outlines the software architecture behind the scientfic workflow systems  and the underlying framework.  These systems are made up of a number of containers, which themselves are made up of a number of components, which in turn are implemented by one or more classes. The architecture for released and planned SimCenter research applications is documented using the C4 model for visualizing the software architecture. A C4 model describes the software architecture in a series of diagrams with four levels of complexity:
\begin{enumerate}
\item A high level diagram showing how software system interacts with the world.
\item Diagrams showing the containers of the system.
\item Diagrams showing the components of each container.
\item Diagrams showing the how the components are built, class diagrams in object-oriented programs.
\end{enumerate}

This document is broken into a number of chapters. In the first we present the software architecture fo the applications EE-UQ, We-UQ, and PBE, which are applications for calculating response for a single structure. The architecture is also presented for RDT, an application for regional loss estimation. In the second we present the architecture behind applications used to generate information used in the regional loss estimation tool RDT. Finally a short descriotion of the versioing numbering employed by SimCenter for its applications. 



\chapter{Installation Instructions}
\label{chap:installation}
All SimCenter applications are available at
the \href{https://simcenter.designsafe-ci.org/research-tools/overview/}{SimCenter
website} under \emph{Research Tools}. The following sections outline
the steps necessary to download and install the \texttt{\getsoftwarename{}}
application. The SimCenter applications do require that you install a
number of other applications that are needed to run the workflows on
your local machine as well as at DesignSafe. \\


%===============================================================================
\section{Download the Application}
%===============================================================================

% \subsection{Download the Application Files}

To download the \texttt{\getsoftwarename{}} application navigate to
the \getsoftwarepage{\texttt{\getsoftwarename{}} page} and click on
the \emph{Download App \& User Manual} link on the right side of the
page. As shown in \Cref{fig:app_choose_file}, this will bring you to another page which contains a list of downloadable files and directories.

\softwareSwitch{PBE}{
\begin{figure}[!htbp]
  \centering {
    \includegraphics[width=0.95\textwidth]
    {installation/figures/pbeDownload.png} }
  \caption{Download Application}
  \label{fig:app_choose_file}
\end{figure}
}{}

\softwareSwitch{EE-UQ}{
\begin{figure}[!htbp]
  \centering {
    \includegraphics[width=0.95\textwidth]
    {installation/figures/eeDownload.png} }
  \caption{Download Application}
  \label{fig:app_choose_file}
\end{figure}
}{}

\softwareSwitch{WE-UQ}{
\begin{figure}[!htbp]
  \centering {
    \includegraphics[width=0.95\textwidth]
    {installation/figures/eeDownload.png} }
  \caption{Download Application}
  \label{fig:app_choose_file}
\end{figure}
}{}


There are at least four files available for download from this page: 
\begin{enumerate}
    \item The PDF file is the User Manual that you are reading now.
    \item The MOV file is an video that provides an introduction to the usage of the application.
    \item The ZIP file is an archive that contains the application files for a Windows operating system.
    \item The DMG file is an archive that contains the application files for a Mac OS X operating system.
\end{enumerate}

To download the \texttt{\getsoftwarename{}} application click on the link for
the appropriate file for your operating system and then click on the
Download button at bottom right corner of the ensuing pop-up window. 
Unpackage the application from the downloaded
file and place it in a location on your filesystem. On Windows, we
recommend that you create a \texttt{C:/SimCenter/\getsoftwarename{}}
directory and extract the contents of the \texttt{ZIP} archive
there. It is also recommended to run the included installer for Visual C/C++ runtime library(vc\_redist.x64.exe). 
If you use a Mac we recommend you copy the application to either your
home folder or your Desktop folder. You are free to place the
applications anywhere you wish, you will need to make the
appropriate adjustments with the following instructions if you do so. \\

Now test that the application starts. To do this navigate to
the location where you placed the application and open it. You should
see the user interface (UI) shown in \Cref{fig:app_UI} after
starting the application. Now Quit the application. Additional steps are required before 
computations can be performed.\\

\softwareSwitch{PBE}{
\begin{figure}[!htbp]
  \centering {
    \includegraphics[width=0.95\textwidth]
    {installation/figures/PBE.png} }
  \caption{PBE Application on Startup}
  \label{fig:app_UI}
\end{figure}
}{}

\softwareSwitch{EE-UQ}{
\begin{figure}[!htbp]
  \centering {
    \includegraphics[width=0.95\textwidth]
    {installation/figures/EE-UQ.png} }
  \caption{EE-UQ Application on Startup}
  \label{fig:app_UI}
\end{figure}
}{}

\softwareSwitch{WE-UQ}{
\begin{figure}[!htbp]
  \centering {
    \includegraphics[width=0.95\textwidth]
    {installation/figures/WE-UQ.png} }
  \caption{WE-UQ Application on Startup}
  \label{fig:app_UI}
\end{figure}
}{}


\begin{enumerate}
\item The SimCenter is not recognized as either a Windows or an Apple vendor. Our applications are not recognized by the operating system as being signed. Consequently, you may receive a warning message when you start the \texttt{\getsoftwarename{}} application for the first time.
\item  On a Mac you will need to right click on the .dmg file to open it. The UI will not start correctly while in the DMG file, you need to open the .dmg file and then copy the \texttt{\getsoftwarename{}} application to your Documents or Desktop folder. You can then move the .dmg file to the trash or eject it after this has been done.
\item  The \texttt{\getsoftwarename{}} application requires additional software outlined in next subsections to work properly. Even of the software starts correctly, it will not run correctly until this software, outlined in the next section, is installed correctly.
\end{enumerate}


%===============================================================================
\section{Set up Python}
%===============================================================================

The SimCenter workflow applications are managed by Python
scripts. These are required to prepare the input data for running
analyses either remotely on DesignSafe or locally. As a consequence the user must have Python
installed on their machine and have the appropriate environment
variables set so that the UI can run these applications.

\subsection{Install Python}

SimCenter products require Python version 3.7 be installed on your machine as January 2020 marks the end of life for Python 2.7. 

\begin{enumerate}
\item Windows:

If you have not yet installed Python 3.7, we recommend installing from \href{https://www.python.org/downloads/windows}{Python.org}. We recommend installing using the \texttt{Windows x86-64 executable installer}. NOTE: At time of writing Python 3.8.0 release will fail to install scipy on Windows 10 and should thus not be installed.

Allow the installer to change your system environment variables so that the install directory is added to your PATH. Once installed you need to You need to install the following python packages: \texttt{numpy}, \texttt{scipy}, and \texttt{pandas} are installed. To install these packages open a \href{https://www.howtogeek.com/194041/how-to-open-the-command-prompt-as-administrator-in-windows-8.1/}{terminal window as an Admin user} and in that window type the following instructions:

To install these packages, start a terminal window and type:

\begin{verbatim}
pip install numpy
pip install scipy
pip install pandas
\end{verbatim}

\item Mac

The Mac comes with Python pre-installed, which is currently the somewhat 
dated version 2.7. To install Python 3.7 we recommend installing from 
\href{https://www.python.org/downloads/}{Python.org}. We recommend installing using the 
\texttt {macOS 64-bit installer} given for latest stable release. The installer will place a python3 executable in your /usr/local/bin directory, whose location should be on your system PATH.

You need to install the following python packages: \texttt{numpy}, \texttt{scipy}, and \texttt{pandas} are installed. 
To install these packages, start a terminal window and type:

\begin{verbatim}
pip3 install numpy
pip3 install scipy
pip3 install pandas
\end{verbatim}

Notes: 
\begin{enumerate}
\item To start a terminal window you can use the spotlight app (magnifying top right of desktop). Start the spotlight app and type in terminal. The terminal application should appear as the top hit. Click on it to start it.
\item In tool preferences make sure that python3 appears as the python executable. If you used older versions of SImCEnter tools this was the default.
\end{enumerate}
\end{enumerate}

\subsection{Test Python}
%===============================================================================

Test if the python environment is set up properly by
executing \texttt{python} in a terminal window. After Python starts,
test if the packages are installed by executing \texttt{import
numpy}, \texttt{import scipy}, and \texttt{import pandas}. You will
receive an error message if a pacakage is missing. If no error
appears, the terminal should look similar
to \Cref{fig:python_test}. Exit Python by executing
the \texttt{exit()} command.

\begin{figure}[!htbp]
  \centering {
    \includegraphics[width=0.8\textwidth]
    {installation/figures/python_test.png} }
  \caption{Testing the Python environment.}
  \label{fig:python_test}
\end{figure}

%===============================================================================
\section{Set up for Running Workflows Locally}\label{setup}
%===============================================================================

To run the workflows locally, the backend python application needs
publicly available software to also be installed on your
machine. These software applications need to be installed and
configured on your operating system. If you do not plan to run the
workflows locally, you will not need these applications.

\subsection{Install \texttt{OpenSees}}
%===============================================================================

\href{http://opensees.berkeley.edu}{\texttt{OpenSees}} is an open-source finite element application publicly available for download from its \href{http://opensees.berkeley.edu/OpenSees/user/download.php}{download page}. \texttt{OpenSees} installation requires the user install both \texttt{OpenSees} and \texttt{Tcl}.  If you have never downloaded \texttt{OpenSees} before, you will need to register your e-mail to gain access. After registration, you can proceed to the download page by entering your email address and clicking the Submit button. The Windows and Mac downloads are in different locations on the download page, with the appropriate Tcl installer beside the \texttt{OpenSees} link; see \Cref{fig:openseesDownload}

\begin{figure}[!htbp]
  \centering {
    \includegraphics[width=\textwidth]
    {installation/figures/openseesDownload.png} }
  \caption{Downloading OpenSees}
  \label{fig:openseesDownload}
\end{figure}

Follow the instructions on the download page to install \texttt{Tcl}
(\Cref{fig:openseesDownload}). On Windows, you must select the Custom option for installton and you must specify
the installtion directory as \texttt{C:\textbackslash Program Files\textbackslash Tcl}, 
which is not the default. \\

After \texttt{Tcl} is installed, we recommend you put \texttt{OpenSees} in
the \texttt{C:/SimCenter/OpenSees} folder on Windows and in
a \texttt{/usr/local/OpenSees} directory on the Mac (If you use finder
on Mac to do navigation, use command-shift-G in Finder and specify
/usr/local as the folder to go to. Create a new folder \texttt{OpenSees}
and copy the \texttt{OpenSees} application to this folder).\\


Now you need to add the \texttt{OpenSees} folder to the
system \texttt{PATH} environment variable to allow the SimCenter
workflow applications to find the \texttt{OpenSees} executable on your
computer. The steps to do this depend on your operating system:

\begin{enumerate}
\item Windows: To add a folder to the \texttt{PATH} on Windows (\Cref{fig:add_env_path}):

\begin{figure}[!htbp]
  \centering {
    \includegraphics[width=0.8\textwidth]
    {installation/figures/add_env_path.png} }
  \caption{Adding OpenSees to the PATH environment variable on Windows}
  \label{fig:add_env_path}
\end{figure}


\begin{enumerate}
    \item open \emph{Start}, type \emph{env}, and choose \emph{Edit the system environment variables};
    \item click on the \emph{Environment variables...} button in the dialog window;
    \item find the \texttt{Path} under \emph{System Variables} in the \emph{Variable} column;
    \item click \emph{New} and type in the path to your \texttt{OpenSees.exe} (this will be \texttt{C:\textbackslash SimCenter\textbackslash OpenSees} if you put the executable at the recommended location - pay attention to using backslashes here!);
    \item click \emph{OK} in every dialog to close them and save your changes.
\end{enumerate}

\item MacOS: To add the /usr/local/OpenSees folder to the \texttt{PATH} variable:

\begin{enumerate}
    \item open a Terminal;
    \item execute (type the following in the terminal window and hit the return key) the following: \begin{verbatim}nano ${HOME}/.bash_profile\end{verbatim}
    \item if the file contains nothing, add the first 3 lines shown in \Cref{fig:add_env_path_Mac} to the file. This is done in
case an existing .bashrc file exists for your system. Adding these 3 lines will test for the existance of this file, and source in any existing commands if the file does exist.
    \item on a new line add the \texttt{OpenSees} executable to the PATH variable, by typing the following: \begin{verbatim}export PATH=/usr/local/OpenSees:${PATH}\end{verbatim}
    \item quit by hitting \texttt{Ctrl+X} and then \texttt{Y} when asked if you want to save modifications.
    \item test it is entered correctl, the following command now entered in the terminal window should result in no errors: \begin{verbatim}source ${HOME}/.bash_profile\end{verbatim}. 
\end{enumerate}

\begin{figure}[!htbp]
  \centering {
     \includegraphics[width=0.8\textwidth]
    {installation/figures/add_env_path_Mac.png} }
  \caption{Adding OpenSees to the PATH environment variable on Mac.}
  \label{fig:add_env_path_Mac}
\end{figure}


\end{enumerate}


\subsection{Install \texttt{Dakota}}
%===============================================================================

\href{http://dakota.sandia.gov}{\texttt{Dakota}}, an open-source  optimization and UQ application from Sandia National Labs, is publicly available for download at its \href{http://dakota.sandia.gov/download.html}{download page}. Select your operating system from the list and set the other options as shown in  \Cref{fig:dakota_installation}. Download the release in a \texttt{ZIP} file for Windows and \texttt{TAR.GZ} file for Mac. We recommend you to extract the archive to a \texttt{C:/SimCenter/Dakota} folder on Windows, and to a \texttt{/usr/local/Dakota} folder on a Mac.

\begin{figure}[!htbp]
  \centering {
    \includegraphics[width=\textwidth]
    {installation/figures/dakota_installation.png} }
  \caption{Downloading Dakota Software}
  \label{fig:dakota_installation}
\end{figure}


Following the instructions provided for installing \texttt{OpenSees}, you need to add \textbf{two} \texttt{Dakota} folders to the system \texttt{PATH} environment variable to allow the SimCenter workflow applications to find the \texttt{Dakota} tools on your computer. 
the procedure described above for \texttt{OpenSees} to add the following
folders to your \texttt{PATH}:

\begin{enumerate}
\item{Windows}

Add the following 2 folders to your windows PATH variable:
\begin{itemize}
    \item \texttt{C:\textbackslash SimCenter\textbackslash Dakota\textbackslash bin}
    \item \texttt{C:\textbackslash SimCenter\textbackslash Dakota\textbackslash share\textbackslash dakota\textbackslash Python}
\end{itemize}

Now you need to create a new variable, \texttt{PYTHONPATH}, and point it to the following folder.


\begin{itemize}
    \item \texttt{C:\textbackslash SimCenter\textbackslash Dakota\textbackslash share\textbackslash dakota\textbackslash Python}
\end{itemize}

\item{MacOS}
On the Mac you also need to add 2 lines, previously shown in \Cref{fig:add_env_path_Mac},
 to the .bash\_profile file. One line adds the Dakota executable to the PATH variablem and 
the other creates a new variable PYTHONPATH and points it to a folder in the  Dakota 
installation directory. 

\begin{itemize}
    \item \texttt{export PATH=/usr/local/Dakota/bin:\${PATH}}
    \item \texttt{export PYTHONPATH=/usr/local/Dakota/share/dakota/Python}
\end{itemize}
\end{enumerate}

NOTE: Apple, in the latest release of their operating system, MacOS 10.16 Catalina, has changed the default working of Gatekeeper.
Gatekeeper, first introduced in OS X Mountain Lion, is a Mac security feature that helps protect your Mac from Malware and other malicious software. Gatekeeper checks to make sure the application is safe to run by checking it against the list of apps that Apple has vetted and approved for the Apple Mac Store and/or approved by Apple even if not offered through the app store. In previous versions of MacOS, Gatekeeper had three security level options: App Store, App Store and Identified Developers, and Anywhere. Anywhere has been removed and this will cause problems with Dakota. As a consequence, it is necessary to follow the following when you update the MacOS or install Dakota for the first time on machine with an updated MacOS. From the terminal app, with the above .bash\_profile settings set, you need to type the following in the terminal window:

\begin{verbatim}
sudo spctl --master-disable
dakota
sudo spctl --master-enable
\end{verbatim}

This will temporarily disable gatekeeper (basically setting Gatekeeper options to Anywhere), allow the Dakota application and it's .dylib files to be registered as safe, and then turn Gatekeeper options back to default.


\subsection{Install  Perl}
%===============================================================================

Mac OS X has Perl pre-installed, but Windows users will have to
install it to be able to use \texttt{Dakota}. We recommend you use Strawberry
Perl; you can install it by downloading the executable from
its \href{http://strawberryperl.com}{Strawberry Perl website} and
running it.

\subsection{Test the Install of the Local Applications}
%===============================================================================

Before running the \texttt{\getsoftwarename{}} application, perform the following tests to
make sure that the local SimCenter working environment is set up
appropriately:

\begin{itemize}
    \item Start a Terminal on Mac or a Command Prompt on Windows.
    \item On Mac, execute \texttt{cd /usr/Documents} to change the active directory to \texttt{/usr/Documents}. On Windows, execute \texttt{cd C:/} to change the active directory to \texttt{C:/}.
    \item Test if \texttt{OpenSees} works correctly by executing the \texttt{OpenSees} command. The command should start \texttt{OpenSees} (\Cref{fig:opensees_test}). Close \texttt{OpenSees} with the \texttt{exit} command.
    \item Test if \texttt{Dakota} works correctly by executing the \texttt{dakota} command. The command should start \texttt{Dakota} and you should see a message about a missing argument (\Cref{fig:dakota_test}).
    \item Test if Perl works correctly by executing the \texttt{perl -v} command. The command should start Perl and return its version number (\Cref{fig:perl_test}).
    \item Test if the python package in \texttt{Dakota} works correctly by starting Python with the \texttt{python} command and then executing the \texttt{import dakota} command. This should import the dakota package. If you do not see errors, then the package is successfully imported (\Cref{fig:dakota_py_test}). Exit Python with the \texttt{exit()} command.
    \item If all the above tests ran without errors, your environment is set up appropriately.
\end{itemize}

\begin{figure}[!htbp]
  \centering {
    \includegraphics[width=0.8\textwidth]
    {installation/figures/opensees_test.png} }
  \caption{Testing OpenSees.}
  \label{fig:opensees_test}
\end{figure}

\begin{figure}[!htbp]
  \centering {
    \includegraphics[width=0.8\textwidth]
    {installation/figures/dakota_test.png} }
  \caption{Testing Dakota.}
  \label{fig:dakota_test}
\end{figure}

\begin{figure}[!htbp]
  \centering {
    \includegraphics[width=0.8\textwidth]
    {installation/figures/perl_test.png} }
  \caption{Testing Perl.}
  \label{fig:perl_test}
\end{figure}

\begin{figure}[!htbp]
  \centering {
    \includegraphics[width=0.8\textwidth]
    {installation/figures/dakota_py_test.png} }
  \caption{Testing the dakota Python package.}
  \label{fig:dakota_py_test}
\end{figure}

%===============================================================================
\clearpage
\section{Test the \texttt{\getsoftwarename{}} application}
\label{sec:test_local}


\softwareSwitch{WE-UQ}{
Once the local SimCenter working environment has been tested and is
functioning correctly, the \texttt{\getsoftwarename{}} Application
can be tested. The simplest way to do this is by running an analysis
using the default structural model with synthetic wind event. By
doing this, it is not necessary to enter any information on the
structural model and only inputs for the synthetic wind forces 
and uncertainty quantification are required. With this quick setup, the
functionality of the \texttt{\getsoftwarename{}} UI and the associated backend
workflow can be tested. The necessary steps to perform this
testing are provided below.

A full description of how to use this software is provided
in \Cref{chap:usage}.  In this quick test, users will only
interface with the event tab (\texttt{EVT}), the uncertainty
quantification tab (\texttt{UQ}), and the results tab (\texttt{RES}).

The first step is to start the \texttt{\getsoftwarename{}}
application.  Once the application is started, the second step is to
input the parameters for the synthetic motions under the \texttt{EVT}
(Event) tab. This is shown in \Cref{fig:input_event}. Click
on the \texttt{EVT} tab which will allow the loading type to be
selected. From the dropdown menu select \texttt{Stochastic Wind Event}
model will be set as \texttt{Wittag and Sinha (1975)}.

\begin{figure}[!htbp]
  \centering {
    \includegraphics[width=0.7\textwidth]
    {installation/figures/testWE_EVT.png} }
  \caption{Selecting event type and inputting synthetic motion parameters}
  \label{fig:input_event}
\end{figure}

Only three inputs are required for this test, of which one will be set
to a random variable. As shown in \Cref{fig:input_event}, set the
\emph{Drag Coefficient} to {\texttt 1.5}, \emph{exposure condition} to \texttt{B}, and \emph{Gust Wind Speed} to \texttt{GWS}.
to \texttt{vs30}. The \texttt{Provide seed value} radio button should
be left unselected. By specifying these inputs, both drag coefficient
and the exposure condition will have constant values in all
realizations while \texttt{GWS} will have different values based on
the model parameters specified in the uncertainty quantification
(\texttt{UQ}) tab. With these inputs specified, navigate to
the \texttt{UQ} tab. Here the distributions and their relevant
parameters will be specified for the random variables defined in the
analysis\textemdash only $GWS$ in this case. Since $GWS$ was
identified as a random variable by inputting the parameter value as
text, it is automatically added as a random variable, as shown
in \Cref{fig:input_uq}. Set the distribution type to \texttt{normal}
with a \texttt{Mean} and \texttt{Standard Dev} of 100 and 10 mph,
respectively.

\begin{figure}[!htbp]
  \centering {
    \includegraphics[width=0.7\textwidth]
    {installation/figures/testWE_UQ.png} }
  \caption{Specifying distribution type and parameters for random
  variables in analysis\textemdash only $GWS$ in this case}
  \label{fig:input_uq}
\end{figure}

Now, click on the \texttt{RUN} button, which will bring up a pop-up
menu that provides information on the application directory and
the working directory. The application directory should already be
automatically set to where \texttt{\getsoftwarename{}} is installed.
If desired, the working directory can be changed. In order to start
the analysis, click on the \texttt{Submit} button.

\begin{figure}[!htbp]
  \centering {
    \includegraphics[width=0.7\textwidth]
    {installation/figures/testWE_RES.png} }
  \caption{Results for test analysis. This tab will open automatically
  when the analysis completes, indicating a successful installation}
  \label{fig:show_results}
\end{figure}



If successful, the application will pause briefly while it runs the
analysis before automatically displaying the simulations results in
the \texttt{RES} tab, as shown
in \Cref{fig:show_results}. Remember, the results shown
in \Cref{fig:show_results} most likely will not be the same as
those from this local test since $GWS$ is a random variable and
the values realized in the simulations will be different while still
following the same distribution. In any case, if the simulations
completed and the \texttt{RES} tab is showing simulation results, then
the \texttt{\getsoftwarename{}} App is properly installed and configured.

}{
Once the local SimCenter working environment has been tested and is
functioning correctly, the \texttt{\getsoftwarename{}} Application
can be tested. The simplest way to do this is by running an analysis
using the default structural model with synthetic ground motions. By
doing this, it is not necessary to enter any information on the
structural model and only inputs for 
\softwareSwitch{PBE}{
the synthetic motions, uncertainty quantification, and loss assessment
}{
the synthetic motions and uncertainty quantification
}
are required. With this quick setup, the
functionality of the \texttt{\getsoftwarename{}} App and the backend
workflow can be tested. The necessary steps to perform this
testing are provided below.

A full description of how to use this software is provided
in \Cref{chap:usage}.  In this quick test, users will only
interface with the event tab (\texttt{EVT}), the uncertainty
quantification tab (\texttt{UQ}), 
\softwareSwitch{PBE}{
the damage and loss assessment tab (\texttt{CMP}),
}{}
and the results tab (\texttt{RES}).

The first step is to start the \texttt{\getsoftwarename{}}
application.  Once the application started, the second step is to
input the parameters for the synthetic motions under the \texttt{EVT}
(Event) tab. This is shown in \Cref{fig:input_event}. Click
on the \texttt{EVT} tab which will allow the loading type to be
selected. From the dropdown menu, as shown
in \Cref{fig:input_event}, select \texttt{Stochastic Ground Motion
Model}. Upon selecting this loading type, the loading model will be
set as \texttt{Vlachos et al. (2018)}.

\softwareSwitch{PBE}{
\begin{figure}[!htbp]
  \centering {
    \includegraphics[width=0.7\textwidth]
    {installation/figures/test_input_event.png} }
  \caption{Selecting event type and inputting synthetic motion parameters}
  \label{fig:input_event}
\end{figure}
}{
\begin{figure}[!htbp]
  \centering {
    \includegraphics[width=0.7\textwidth]
    {installation/figures/test_input_event.png} }
  \caption{Selecting event type and inputting synthetic motion parameters}
  \label{fig:input_event}
\end{figure}
}

Only three inputs are required for this test, of which one will be set
to a random variable. As shown in \Cref{fig:input_event}, set the
\emph{Moment Magnitude} ($M_W$) to 6.5, the \emph{Closest-to-Site Rupture Distance}
($R_{rupt}$) to 20 km, and the \emph{Average shear-wave velocity
for the top 30 m} ($V_{S_{30}}$)
to \texttt{vs30}. The \texttt{Provide seed value} radio button should
be left unselected. By specifying these inputs, both $M_{W}$ and $R_{rupt}$
will have constant values in all realizations while $V_{S_{30}}$ will
have different values based on the model parameters specified in the
uncertainty quantification (\texttt{UQ}) tab.

With these inputs specified, navigate to the \texttt{UQ} tab. Here the
distributions and their relevant parameters will be specified for the
random variables defined in the analysis\textemdash only $V_{S_{30}}$
in this case. Since $V_{S_{30}}$ was identified as a random variable
by inputting the parameter value as text, it is automatically added as
a random variable, as shown in \Cref{fig:input_uq}. Set the
distribution type to \texttt{normal} with a \texttt{Mean}
and \texttt{Standard Dev} of 350 m/s and 25 m/s,
respectively.

\softwareSwitch{PBE}{
\begin{figure}[!htbp]
  \centering {
    \includegraphics[width=0.7\textwidth]
    {installation/figures/test_input_uq.png} }
  \caption{Specifying distribution type and parameters for random
  variables in analysis\textemdash only $V_{s30}$ in this case}
  \label{fig:input_uq}
\end{figure}
}{
\begin{figure}[!htbp]
  \centering {
    \includegraphics[width=0.7\textwidth]
    {installation/figures/test_input_uq.png} }
  \caption{Specifying distribution type and parameters for random
  variables in analysis\textemdash only $V_{s30}$ in this case}
  \label{fig:input_uq}
\end{figure}
}

\softwareSwitch{PBE}{
The last step before running the analysis is to set up a damage and loss model. Navigate to the \texttt{CMP} tab and select \texttt{HAZUS MH} as the loss assessment method to use. Set up the model according to \Cref{fig:input_dl}.
\begin{figure}[!htbp]
  \centering {
    \includegraphics[width=0.7\textwidth]
    {installation/figures/test_input_uq.png} }
  \caption{Specifying the damage and loss model for the analysis.}
  \label{fig:input_dl}
\end{figure}
}{}

Now, click on the \texttt{RUN} button, which will bring up a pop-up
menu that provides information on the application directory and
the working directory. The application directory should already be
automatically set to where \texttt{\getsoftwarename{}} is installed.
If desired, the working directory can be changed. In order to start
the analysis, click on the \texttt{Submit} button.

\softwareSwitch{PBE}{
\begin{figure}[!htbp]
  \centering {
    \includegraphics[width=0.7\textwidth]
    {installation/figures/test_uq_res.png} }
  \caption{Results for test analysis. This tab will open automatically
  when the analysis completes, indicating a successful installation}
  \label{fig:show_results}
\end{figure}
}{
\begin{figure}[!htbp]
  \centering {
    \includegraphics[width=0.7\textwidth]
    {installation/figures/test_uq_res.png} }
  \caption{Results for test analysis. This tab will open automatically
  when the analysis completes, indicating a successful installation}
  \label{fig:show_results}
\end{figure}
}

If successful, the application will pause briefly while it runs the
analysis before automatically displaying the simulations results in
the \texttt{RES} tab, as shown
in \Cref{fig:show_results}. Remember, the results shown
in \Cref{fig:show_results} most likely will not be the same as
those from this local test since $V_{S_{30}}$ is a random variable and
the values realized in the simulations will be different while still
following the same distribution. In any case, if the simulations
completed and the \texttt{RES} tab is showing simulation results, then
the \texttt{\getsoftwarename{}} App is properly installed and configured.

}
%===============================================================================


\chapter{Usage}
\label{chap:usage}
\section{User Interface}
The user interface (UI), as shown in \Cref{fig:figure1}, is where the analysis
is configured and managed. Here, the user is able to provide the necessary
parameters to create the simulation, start the simulation both locally and
remotely, and view the simulation results. The interface contains several
separate areas:

\begin{figure}[!htbp]
  \centering {
    \includegraphics[width=0.8\textwidth]
    {usage/figures/ui.png} }
  \caption{UI}
  \label{fig:figure1}
\end{figure}

\begin{enumerate}
\item Input Panel Selection: This area on the left side provides the
  user with a selection of items to choose from from:
\begin{enumerate}
  \item GI: General Information (\ref{sec:generalInfo}), for specifiction of building
    description, location and units.
  \item SIM: Structure Information Model (\ref{sec:structuralInfo}), for description of the
    building model.
  \item EVT: Event (\ref{sec:event}), for selecting the input earthquake motions for building.
  \item FEM: Finite element method (\ref{sec:fem}), for specifying the analysis options.
  \item UQ: Uncertainty quantification (\ref{sec:uq}), for defining the distribution
    of the random variable paramaters and UQ method analysis options.
  \item EDP: Engineering Demand Parameters (\ref{sec:edp}), for specification of
    output response quantities.
  \item RES: Results output (\ref{sec:results}), for looking at the results.
\end{enumerate}

Selecting any of these will change the input panel presented.

\item Input Panel: This is the large central area of the UI that the
  user provides input for the application chosen and views the
  results. For example if the user had selected UQ in the input panel
  selection, it is in this panel that the user would provide details
  on the distributions associated with each random variable or select
  the sampling method to use and provide the options necessary to run
  that method.

\item Push Buttons: This is the area near the bottom of the UI in
  which 4 buttons are presented to the user:

\begin{enumerate}
\item RUN – to run the simulation of the user’s desktop machine.
\item RUN at DesignSafe – to process the information, and send to
  DesignSafe where the job will be run on a supercomputer and results
  stored in your DesignSafe jobs folder.
\item GET from DesignSafe – to obtain from DesignSafe your list of
  jobs and select from that list a job to download.
\item Exit: to exit the application.
\end{enumerate}

The use of the push buttons is discussed in \Cref{sec:push_buttons}.

\item Login Button: At the top right of the UI is the login
  button. Before the user can launch any jobs on DesignSafe, they must
  first login to DesignSafe using their DesignSafe login and
  password. Pressing the login button will open up the login window
  for users to enter this information. Users can register for an
  account on
  the \href{https://www.designsafe-ci.org/account/register/}{DesignSafe
  webpage}.

\item Message Area: In the top center of the application is the area
  of the interface that error and status messaged will be displayed
  while the application is running.

\end{enumerate}


\section{GI: General Information}
The user here provides information about the building and the units
the user will work with. The widget contains 4 separate frames,
as shown in \Cref{fig:gi_overview}:

\begin{enumerate}
\item Building Information: Collects general information about the building, including year of construction and type of structure.
\item Properties: Collects information about number of stories, width, depth, plan area and height of the building.
\item Location: Collects information about the location of the building. This information is used in some event widgets to obtain events specific to the building location.
\item Units: Collects information about the units for the inputs and outputs. Some widgets will require inputs in different units. Those widgets will display units beside those special entry fields.
\end{enumerate}

\begin{figure}[!htbp]
  \centering {
    \includegraphics[width=0.5\textwidth]
    {usage/figures/gi.png} }
  \caption{General Information Input Panel}
  \label{fig:gi_overview}
\end{figure}



\section{SIM: Structural Information Model}
This panel is where the user defines the structural model of the
building. The structural model is that part of the building provided
to resist the lateral loads. There are a number of backend
applications provided for this part of the workflow, each responsible
for providing the structural analysis model to the workflow. The
drop-down menu at the top of this panel is where the user selects
which application to use. As the user switches between applications,
the entry data changes to reflect the different inputs the different
applications require. At present there are two backend applications
available through the drop down menu: Multiple Degrees of Freedom
(MDOF) and OpenSees.

\subsection{Multiple Degrees of Freedom (MDOF)}

This panel is provided for users to quickly create simple shear models
of a building. The panel, as shown in \autoref{fig:mdof} is divided
into 3 frames:
\begin{enumerate}
\item in the top left frame the user enters the number of stories. For
  each story the user then enter the story height, initial stifness in
  1 and 2 directions, yield strength in 1 and 2 directions, and
  hardeing ratio again in 1 and 2 directions. In addition user enters
  the floor weights and damping ratios for each of the modes.
\item In the lower left frame the user has the option of overriding an
  an individual floor or story basis, any of the properties set in the
  upper frame.
\item on the right side of the frame is a graphical widget showing the
  current building. When entering data into the lower left frame,
  those floors and stories corresponding to data being modified is
  highlighted in red.
\end{enumerate}

\begin{figure}[!htbp]
  \centering {
    \includegraphics[width=0.8\textwidth]
    {./usage/figures/mdof.png} }
  \caption{MDOF Model Definition}
  \label{fig:mdof}
\end{figure}

Random Variables: Random Variables can be created by the user entering
a valid string instead of a number in the entry fields for all entries
except the number of floors. The variable name entered will appear as
a Random Variable in the UQ panel and it is there that the user must
enter the distribution for the Random Variable.


\subsection{OpenSees}
The panel is for users who have existing OpenSees model of a building
that performs a gravity analysis and now wish to subject that building
model to one of the EVENT options provided. The panel that presents is
as shown in \autoref{fig:figure3}. The user has 3 fields that he needs
to fill out:
\begin{enumerate} 
\item The user specifies the main script that contains the building
  model. This script should build a model and perform any gravity
  analysis of the building that is required before the event is
  applied.
\item A llist of nodes that define a column line of interest for which
  the responses will be determined. The column nodes should be in
  order from ground floor through to roof. The EDP options use this
  information to determine nodes at which displacement, acceleration
  and story drifts are calculated.
\item An entry for the dimension of the model, i.e. 1D, 2D or 3D. This
  information is used to apply ground motions.
\end{enumerate}

\begin{figure}[!htbp]
  \centering {
    \includegraphics[width=0.8\textwidth]
    {./usage/figures/openSees.png} }
  \caption{OpenSees Input Model}
  \label{fig:figure3}
\end{figure}

Random Variables: In OpenSees there is an option to set variables to
have certain values using the pset command, e.g pset a 5.0 will set
the variable a to have a value 5 in an OpenSees script. In EE-UQ any
variable found in the main script to be set using the pset command
will be assumed to be a Random Variable. As such, when a new main
script is loaded all variables set with pset will appear as Random
Variables in the UQ panel.


\section{EVT: Event}
The event panel presents the user with a drop-down menu with a list of
available event applications. Event applications are applications
that, given the building and user supplied data inputs, will generate
a list of events for the building. There are a number of options
available in the drop-down menu, as described below.

\subsection{Multiple Existing}
\label{subsec:multi_existing}
This is provided for the user to specify multiple existing SimCenter
Event files.  If more than one event is specified it is done to provide
the UQ engine with a discrete set of events to choose from.  It is not
done with the intention of specifying that one event follows another.
The panel presented initially to the user is as shown
in \autoref{fig:figure4}.

\begin{figure}[!htbp]
  \centering {
    \includegraphics[width=0.8\textwidth]
    {usage/figures/multipleExisting1.png} }
  \caption{\texttt{Multiple Existing} events selected as EVT loading type}
  \label{fig:figure4}
\end{figure}

To add a new event, the user presses the Add button.  This adds an event to the panel.  Pressing the button multiple times will keep adding events to the panel.  \autoref{fig:figure5} shows the state after the button has been pressed twice, and data entered for the ElCentro and Rinaldi Events.

\begin{figure}[!htbp]
  \centering {
    \includegraphics[width=0.8\textwidth]
    {usage/figures/multipleExisting2.png} }
  \caption{Adding new event under \texttt{Multiple Existing} loading type}
  \label{fig:figure5}
\end{figure}

The user can enter the full path manually to the file or use the
choose button, which brings up your typical file search screen.  By
default, a scaling factor of 1.0 is assigned to the event.  The user
can change this to another real value (AT PRESENT DO NOT USE INTEGER)
or the user has the option of defining this to be a random variable by
entering a name as shown for the second event.

Note that this variable name must not start with a number, or contain
any spaces or special characters, i.e. no -, +,..

The Remove button is pressed to remove events. To remove an event the
user must first select which events they wish to remove, done by
clicking in the small circle at the start of the event. Once the
events to remove have been selected, the user removes all these
selected evens by pressing the remove button.

If the user has multiple events to load, all the event files may first
be placed by the user into a seperate folder. If the user presses the
Load Directory, the user will be able to choose a directory and the
application will load all the event file (any file with a .json
suffix) into the widget by choosing the directory. Initially each
event will be given a load factor of 1.0.  Should the user include in
that directory a file named \texttt{Records.txt} the application will open that
file and load the events and assigned load factors from that
file. Each line in Recors.txt is considered ro represent a record, and
contains 2 comma seperated values: the first value being the event
file and the second value the event factor. An example \texttt{Records.txt} is
as shown below:

\begin{verbatim}
ElCentro.json,1.5
Rinaldi.json,2.0
\end{verbatim}

Random Variables: The user can, as mentioned, enter a string in the
factor field to specify that the factor is to be considered a random
variable. Subsequently in the UQ panel the user must provide
information on the random variables distribution. Also, if multiple
events are specified, the event itself will be treated as a random
variable, with each event being part of the discrete set of possible
events.


\subsection{Multiple PEER Event}
This event option is provided for the user to specify multiple existing
\href{http://peer.berkeley.edu}{PEER} ground
motion files. PEER files contain time-histories in a single degree of freedom; hence, if multi-degree-of-freedom excitation is desired, the user is required to specify each
individual component for every EVENT. The \texttt{Add/Remove} buttons
at the top are to create and remove an event, as
per \Cref{subsec:multiple_existing}. The \texttt{+} and \texttt{-} buttons add and remove
components (see \Cref{fig:PEER_event_panel}). Remove removes all selected components. Each
component in a PEER event can have their own scale factor, which can be assigned a random variable.

\begin{figure}[!htbp]
  \centering {
    \includegraphics[width=0.8\textwidth]
    {usage/figures/multiplePEER.png} }
  \caption{Multiple PEER Events}
  \label{fig:PEER_event_panel}
\end{figure}

The \texttt{Load Directory} button and the \texttt{Records.txt} file works for PEER events as described in \Cref{subsec:multiple_existing}. The only difference is that PEER files are expected to have an \texttt{.AT2} extension. Only files with that extension shall be specified in the \texttt{Records.txt}.
An example \texttt{Records.txt} file for multiple Peer
events is shown below:

\begin{verbatim}
elCentro.AT2,1.5
Rinaldi228.AT2,2.0
Rinaldi318.AT2,2.0
\end{verbatim}

Random Variables: Random scale factors can be defined by entering a string in the factor field. The variable name entered will appear as a Random Variable in the UQ panel and the user must specify its properties there. If multiple
events are specified, the event itself will be treated as a random
variable, with each event being part of the discrete set of possible
events.


\subsection{Hazard Based Event}
The panel for this event application is as shown in
\autoref{fig:figure7}.  This application implements a scenario-based
(deterministic) seismic event.  In this panel the user specifies an
earthquake rupture (location, geometry and magnitude), a ground motion
prediction equation, a record selection database and the intensity
measure used for record selection.  In the backend, this application
relies on three other applications to perform seismic hazard analysis,
intensity measures simulation (to create a simulated target spectrum),
and ground motion record selection/scaling.  Users interested in
learning about those applications are referred to the documentation of
the
(\href{https://github.com/NHERI-SimCenter/GroundMotionUtilities/blob/master/Readme.md}{SimCenter
  ground motion utilities}).

\begin{figure}[!htbp]
  \centering {
    \includegraphics[width=0.8\textwidth]
    {usage/figures/hazardBased.png} }
  \caption{\texttt{Hazard Based Event} loading type}
  \label{fig:figure7}
\end{figure}


\subsection{Stochastic Ground Motion Model}
This option allows users to generate synthetic ground motions for a
target seismic event. In order to do so, the stochastic ground motion
model is selected from the drop-down menu, as shown
in \Cref{fig:stochastic_loading}. Depending on the model selected, the
user will be asked to enter values for a number of parameters that are
used to generate a seismic event. In the current release, users can
select between the model derived by Vlachos et
al. (2018) \cite{vlachos2018predictive} and the model developed by
Dabaghi \& Der Kiureghian (2014, 2017, 2018)
[\cite{dabaghi2014stochastic}, \cite{dabaghi2017stochastic}, \cite{dabaghi2018simulation}]. The
geometric directivity parameters, as shown in \Cref{fig:dabaghi}, required by the Dabaghi \& Der
Kiureghian model are described completely in Somerville et al. (1997)
\cite{somerville1997modification}. Additionally, users can provide a
seed for the stochastic motion generation if they desire the same
suite of synthetic motions to be generated on multiple occasions.  If
the seed is not specified, a different realization of the time history
will be generated for each run. The backend application that generates
the stochastic ground motions relies on \texttt{smelt}, a modular and
extensible C++ library for generating stochastic time histories. Users
interested in learning more about the implementation and design of
\texttt{smelt} are referred to its
\href{https://github.com/NHERI-SimCenter/smelt}{GitHub repository}.

All input parameters can be specified as random variables by entering
a string in the parameter field. Please note that information for the
inputs that are identified as random variables needs to be provided in
the \texttt{UQ} tab.

\begin{figure}[!htbp]
  \centering
    \begin{subfigure}{\textwidth}
        \centering
        \includegraphics[width=\textwidth]{usage/figures/stochastic_loading.png}
        \caption{Vlachos et al. (2018) model inputs}
    \end{subfigure}
    \hspace*{\fill}
    
    \begin{subfigure}{\textwidth}
        \centering
        \includegraphics[width=\textwidth]{usage/figures/stochastic_dabaghi.png}
        \caption{Dabaghi \& Der Kiureghian (2018) model inputs}
        \label{fig:dabaghi}        
    \end{subfigure}
    
  \caption{Stochastic Ground Motion Event}
  \label{fig:stochastic_loading}    
\end{figure}


\subsection{Site Response}
This option allows users to determine the event at the base of the 
building by performing an effective free-field site response 
analysis of a soil column. In this panel the user specifies a ground 
motion at the bottom of the column. After the soil layers have been properly 
defined, the motion at the ground surface are given at the end 
of the analysis and that motion will be used in the 
simulation of the building response. 

\begin{figure}[!htbp]
  \centering {
    \includegraphics[width=0.8\textwidth]
    {usage/figures/s3hark1.png} }
  \caption{Site Response Analysis Event}
  \label{fig:s3hark1}
\end{figure}

The UI of the Site Response is shown in \Cref{fig:s3hark1}. It is split into the following areas:
\begin{enumerate}
\item Soil Column Graphic: The first graphic on the left of the panel shows a visualization of the soil column.
\item FE Mesh Graphic: The second graphic on the left shows 
the finite element mesh and profile plots. Selecting any of the tabs on the right inside this graphic (i.e, PGA, $\gamma_{max}$, maxDisp, maxRu, maxRuPWP) will show various results
from the simulation at the mesh points.
\item Operations Area: The right side of this area shows some information (e.g., total height and number of soil layers), includes the Ground Water Table (GWT) input field, and plus and minus buttons. If the user presses the plus button, a layer is added below the selected layer. If the minus button is pressed the selected layer is removed. The GWT input field allows the user to specify the level of the ground water table.
\item Soil Layer Table: This table is where the user provides the characteristics of the soil layer, such as layer thickness, density, vs30, material type, and element size in the finite element mesh.
\item Tabbed Area: This area contains the three tabbed widgets described below.

\begin{enumerate}
  \item Configure Tab: This tab allows the user to specify the path to the OpenSees executable and to a ground motion file that represent the ground shaking at the bedrock. The rock motion file must follow the SimCenter event format. Examples of SimCenter event files are available with the \href{https://github.com/NHERI-SimCenter/EE-UQ/tree/master/example1/event}{source code} 
  \item Layer Properties Tab: This tab allows the user to enter additional material properties for the selected soil layer (\Cref{fig:s3hark3}).
  \item Response Tab: Once the site response analysis has been performed, this tab provides information about element and nodal time varying respone quantaties.
  \item Run Tab: Opens up a window in which by using the up and down arrows on the keyboard the dino will jump up and down. Something to do if the site response analysis is taking too long, which it may if many soil layers are used.
\end{enumerate}

\item Analyze Button: This button shall be used to run the simulation locally. A progress bar will show the status of the analysis. This allows the user to review the ground motion predicted at the surface.
\end{enumerate}

\begin{figure}[!htbp]
  \centering {
    \includegraphics[width=0.8\textwidth]
    {usage/figures/s3hark3.pdf} }
  \caption{Soil Layer Modification in Site Response }
  \label{fig:s3hark3}
\end{figure}

Upon the finish of the finite element analysis, the ground motion at the soil surface (\Cref{fig:s3hark4}) will be stored in EE-UQ's input file.
This computed motion will be applied during the simulation.

Random Variables: The current version of the Site Response event type does not support random variables.\\

NOTES: 
\begin{enumerate}
\item Variables are assumed to have m, kPa, and kN units in the Site Response panel.
\item If the Analyze button is not pressed, no simulation will be performed. If no simulation is performed there will be no ground motions provided to the building.
\end{enumerate}


\subsection{User Application}
The final option for event definition is a user application. 
The user specifies the application name and the input file containing the specific input information 
needed by the application when it is running in the backend. 
As will be discussed later, is the user selects to utilize an application that is not provided, they are also required to edit the tools registry file. Here they must include a new event application with the same name as that entered and they must provide the location where that application can be found relative to the tools application directory. 
If running on DesignSafe, that application must be built and must be available on the Stampede2 supercomputer. 

Note: Given how DesignSafe runs the applications through Agave, the file permissions of this application must be world readable and executable.

\begin{figure}[!htbp]
  \centering {
    \includegraphics[width=1.0\textwidth]
    {usage/figures/userAppEvent.png} }
  \caption{User defined event}
  \label{fig:user_defined_event_panel}
\end{figure}



\section{FEM: Finite Element Method}
The FEM panel will present users with a selection of FEM
applications that will take a building model generated by the BIM
application and the EVENT from the event application and perform a
deterministic simulation of structural response. Currently, there is one application
available, OpenSees, and there is no application selection box. That
will be modified in future versions to allow users to provide their own
simulation application. The current OpenSees implementation extends the standard OpenSees executable with a pre- and post-processor to take the BIM and EVENT
files and use OpenSees to simulate the response, and return it in an EDP file.

\begin{figure}[!htbp]
  \centering {
    \includegraphics[width=0.5\textwidth]
    {usage/figures/fem.png} }
  \caption{Options for \texttt{OpenSees} transient analysis}
  \label{fig:fem}
\end{figure}

For the OpenSees application the user is required to specify the
options to be used in the transient analysis. As shown in \Cref{fig:fem},
this includes the choice of
\begin{enumerate}
\item \href{http://opensees.berkeley.edu/wiki/index.php/Algorithm_Command}{Solution algorithm}, the default is Newton Raphson.
\item \href{http://opensees.berkeley.edu/wiki/index.php/Integrator_Command}{Integration Scheme}, the default is Newmark's linear acceleration
  method.
\item \href{http://opensees.berkeley.edu/wiki/index.php/Test_Command}{Convergence Test}, the default is a norm on the unbalance force.
\item Convergence tolerance, with a default of 0.01.
\item Damping Ratio. the default is 2\% equivalent viscous damping entered as 0.02. If
a damping ratio of 0 is specified, no damping is added by the simulation application. This allows users to add their own damping in the OpenSees tcl script they load under SIM.
\item Analysis Script. This shall be left blank by default. Advanced users of OpenSees who have their preferred analysis script
and wish to provide their own damping model can provide it here.
\end{enumerate}


The options available for each setting can be found in the OpenSees online user
manual.\\

A default transient analysis script is run with these inputs. It is
built for Version 3.0.0+ of OpenSees and uses a divide and conquer
algorithm to overcome convergence issues. This new algorithm
does not work for every nonlinear problem. The actual analysis command
that is created based on the defaults is the following:

\begin{verbatim}
numberer RCM
system Umfpack
integrator Newmark 0.5 0.25
test NormUnbalance 0.01 20 
algorithm Newton
analysis Transient -numSubLevels 2 -numSubSteps 10 
analyze $numStep $dt
\end{verbatim}

If the user specifies their own analysis script to run
instead of the default, they can take advantage of the \texttt{numStep} and \texttt{dt} variables that
are obtained from the EVENT and are automatically set by the program.


\section{UQ: Uncertainty Quantification}
Throughout the input specification the user is defining variables. As
described in the above sections many of these variables can be
specified by the user to be random variables with a distribution on
their values. It is in the UQ panel that the user specifies what these
distributions are.
It is also here that the user specifies the UQ method and the input
values are for these UQ methods.  The panel is split, as shown
in \Cref{fig:figure10}, into 2 frames:

\begin{enumerate}
\item Sampling Methods 
\item Random Variables
\end{enumerate}

\begin{figure}[!htbp]
  \centering {
    \includegraphics[width=0.8\textwidth]
    {usage/figures/uq1.png} }
  \caption{Uncertainty Quantification input panel}
  \label{fig:figure10}
\end{figure}

\subsection{Sampling Methods}
In the \href{https://dakota.sandia.gov//sites/default/files/docs/6.9/html-ref/methods-sampling.html}{sampling methods} the user selects the sampling 
method to use from the method dropdown. Currently this is limited to 
two options: Monte Carlo and Latin Hypercube Sampling (LHS). For the 
one selected, the user specifies the number of simulations to be 
perform and the seed.

\subsection{Random Variables}
The Random Variable panel is where the user enters the random
variables. Each random variable has a name and a distribution. The
distribution is selected from the drop-down menu. By changing the
distribution type, the inputs required to define the distribution
change. The following are the list of distributions available (clicking on any link will take you to Dakota manual explaining inputs and theory):
\begin{enumerate}
\item \href{https://dakota.sandia.gov//sites/default/files/docs/6.9/html-ref/variables-normal_uncertain.html}{Normal}
\item \href{https://dakota.sandia.gov//sites/default/files/docs/6.9/html-ref/variables-lognormal_uncertain.html}{Lognormal}
\item \href{https://dakota.sandia.gov//sites/default/files/docs/6.9/html-ref/variables-beta_uncertain.html}{Beta}
\item \href{https://dakota.sandia.gov//sites/default/files/docs/6.9/html-ref/variables-uniform_uncertain.html}{Uniform}
\item \href{https://dakota.sandia.gov//sites/default/files/docs/6.9/html-ref/variables-weibull_uncertain.html}{Weibull}
\item \href{https://dakota.sandia.gov//sites/default/files/docs/6.9/html-ref/variables-gumbell_uncertain.html}{Gumbell}
\end{enumerate} 

As with other panels, the random variables can be added or
removed. Care must be taken by the user in ensuring that if the user
removes random variables from this panel that they also remove them
from the other input widgets. Failing to do so may result in the
program failing to complete.


\section{EDP: Engineering Demand Parameters}
This panel is where the user selects the outputs to be displayed when
the simulation runs. There are two options available in the pull-down
menu:
\begin{enumerate}
\item Standard Earthquake
\item USer Defined
\end{enumerate}

\subsection{Standard Earthquake}
When the user selects standard Earthquake there are no additional
inputs required. The standard earthquake EDP generator will ensure the
the max absolute value of the following are
obtained:
\begin{enumerate}
\item Relative Floor displacements:
\item Absolute Floor Accelerations
\item Interstory Drifts
\end{enumerate}

The results will contain results for these in abbreviated form:
\begin{itemize}
\item PFD peak relative floor displacement $1-PFD-FLOOR_CLINE$
\item PFA peak floor acceleration (relative + ground motion):
  $1-PFA-FLOOR-CLINE$
\item PID peak inter-story drift: $1-PID-STORY-CLINE$
\end{itemize}

\subsection{User Defined}
This panel allows the user to provde to determine their own output and
process it. When using this option the user provides additional data:
\begin{enumerate}
\item Additional Input: These are additional commands that are invoked
  by the analysis application before the transient analysis is
  performed. For example, foe OpenSees this would be a script
  containing a series of recorder commands.
\item Postprocess Script: This is a python script that will be invoked
  after the finite element application has run. It must be provided by
  the user. It's purpose is to process the output files and create a
  single file, results.out. This file must contain a single line with
  as many entries as EDP's specified.
\item Response Parameters. This is an area in which the user
  associates a variable name with the column of the results output
  file. If the process script has an array of strings named named
  EDP's the script, the Response Parameters will be initially set with
  these values from the script.
\end{enumerate}


\section{RES}
When the user hits the Run button, and assuming the results are
successful. The results are presented here.  A successful run or
download of a job that ran successfully will result in 3 tabbed
widgets being displayed in this panel.  The first panel, shown in
\Cref{fig:results_summary} shows summary statistics: mean and
stdDev values or min-max values if discrete set, i.e. multiple events
for each of the EDP's specified in the EDP panel. The second panel,
shown in \Cref{fig:summary_information} shows the summary
information.

\begin{figure}[!htbp]
  \centering {
    \includegraphics[width=0.8\textwidth]
    {usage/figures/resultsSummary.png} }
  \caption{Results Summary}
  \label{fig:results_summary}
\end{figure}

\begin{figure}[!htbp]
  \centering {
    \includegraphics[width=0.8\textwidth]
    {usage/figures/resultsInformation.png} }
  \caption{\texttt{General} tab showing results summary information}
  \label{fig:summary_information}
\end{figure}

The third panel, shown in \Cref{fig:results_data} presents
graphically and in tabular form the results. By selecting different
columns with left and right mouse buttons in the table below the
graphic, the information in the graph is changed. Selecting the left
mouse button changes the Y axis, the right mouse changes the X
axis. If the same column is selected using both left and right keys,
the CDF and PDF is displayed. If last mouse press was with the left
button, the PDF and if right the CDF.
 
As for the columns. You will see a column for each random variable the
workflow came across. There may be more than you specified if the
applications want the UQ engine to consider their own variables in the
computation. The outputs at present are limited to:

\begin{figure}[!htbp]
  \centering {
    \includegraphics[width=0.8\textwidth]
    {usage/figures/resultsData.png} }
  \caption{Results presented graphically and in tabular form}
  \label{fig:results_data}
\end{figure}


\section{Push Buttons}
There are a number of buttons in the Push Button area of \Cref{fig:generic_ui}. This section describes the usage of those buttons.

\subsection{RUN}

This button allows the user to run the simulation on the local machine. The window that pops up is as shown in \Cref{fig:run_button_popup}. There
are two input fields, and a button in the window. The typical user DOES NOT NEED TO MODIFY THE INFORMATION IN THE INPUT FIELDS. Advanced users can specify the locations of the following two directories here:

\begin{figure}[!htbp]
  \centering {
    \includegraphics[width=0.8\textwidth]
    {usage/figures/runButton.png} }
  \caption{Pop-up window shown after clicking on the \texttt{Run} button}
  \label{fig:run_button_popup}
\end{figure}

\begin{itemize}
\item Working Dir Location: specifies where the \texttt{\getsoftwarename{}} application shall
create a \texttt{tmp.SimCenter} directory for temporary files that are used to perform the simulation. This directory is created after the \texttt{Submit} button is pressed. As discussed in \Cref{chap:troubleshooting}, when
the application creates this directory it copies the files needed to it (e.g., if you are using OpenSees input script, it
will copy that script to the \texttt{tmp.SimCenter} directory. ALL FILES IN
THE SCRIPT DIRECTORY AND ALL FILES IN SUBDIRECTORIES OF THAT DIRECTORY GET
COPIED SO DON’T PLACE THE OPENSEES SCRIPT IN HOME, DOWNLOADS, DOCUMENTS, etc….
\item Application Dir Location: The \texttt{\getsoftwarename{}} application searches for the workflow applications in this directory. Only edit its location if you are introducing your own applications or you want to build and modify the 
applications provided with the tool. 
\end{itemize}

Hitting the \texttt{Submit} button starts the simulation by running the various workflow applications. After the simulation finished, the pop-up window will close and the RES panel will be in focus. Please do not press the \texttt{Submit} button multiple times. It will not make the simulation run faster and might cause unexpected behavior.

\subsection{RUN at DesignSafe}
This button allows the user to package the input information, send it to DesignSafe and run the simulation remotely at the Stampede2 supercomputer. Simulation results will be stored in the user's DesignSafe jobs folder and can be retrieved with the \texttt{GET from DesignSafe} button. After clicking on the button, the window shown in \Cref{fig:remote_button} pops up. There are several input fields and a \texttt{Submit} button in the window. The typical user ONLY NEEDS TO EDIT THE TOP FOUR FIELDS. The purpose of the lower four fields is described below for advanced users. The following pieces of information are collected in the pop-up window:

\begin{figure}[!htbp]
  \centering {
    \includegraphics[width=0.8\textwidth]
    {usage/figures/remoteButton.png} }
  \caption{Pop-up window shown after clicking the \texttt{RUN at DesignSafe} button}
  \label{fig:remote_button}
\end{figure}

\begin{itemize}
\item Job Name: The name the user can use to identify the job in Get from DesignSafe.
\item Num Nodes: The number of compute nodes to use on Stampede2. Using the default App Name the job will run on Stampede2’s KNL Landing (KNL) 
compute nodes. Each node has 68 cores. The actual number of cores the
application will use on each of these nodes depends on the total
number of processes specified. As per the TACC webpage, for MPI tasks
it’s best not to specify more than 64-68 processes to run. Depending
on the numerical computations and amount of memory each uses, for large simulations you may wish to use more nodes and less processes to
avoid page faulting.
\item Total Number of Processes: Total number of MPI parallel processes the UQ engine is going to use.
\item Max Wall Time: Use HOURS:MIN:SEC format and be conservative. Your job is killed after the time limit is reached. On Stampede2 you have a max wall time of 24 hours.
\item App Name: Name of Agave app to run. Only modify if you know what you are doing.
\item Working Dir Location: specifies where the \texttt{\getsoftwarename{}} application shall
create a \texttt{tmp.SimCenter} directory for temporary files that are used to perform the simulation. This directory is created after the \texttt{Submit} button is pressed. As discussed in \Cref{chap:troubleshooting}, when
the application creates this directory it copies the files needed to it (e.g., if you are using OpenSees input script, it
will copy that script to the \texttt{tmp.SimCenter} directory. ALL FILES IN
THE SCRIPT DIRECTORY AND ALL FILES IN SUBDIRECTORIES OF THAT DIRECTORY GET
COPIED SO DON’T PLACE THE OPENSEES SCRIPT IN HOME, DOWNLOADS, DOCUMENTS, etc…. The Working Directory is removed after the job has been submitted successfully.
\item Local App Dir Location: The \texttt{\getsoftwarename{}} application searches for the workflow applications in this directory. Only edit its location if you are introducing your own applications or you want to build and modify the 
applications provided with the tool. 
\item Remote App Dir Location: Remote directory on Stampede2 where applications needed by the workflow reside. Only modify if you know what you are doing.

\end{itemize}

\subsection{GET from DesignSafe}
Allows you to obtain your list of jobs from DesignSafe and select from that list a job to update status of, download or delete.

\subsection{Exit}
Click this button to exit the application.


\chapter{Theory and Implementation}
\label{chap:theory}
\input{TheoryAndImplementation.tex}

\chapter{Source Code}
\label{chap:SourceCode}
This source code for the tool is released under the 2-clause BSD License, commonly called the FreeBSD license. 
It is available for download from the tools GitHub repository: \\
\href{https://github.com/NHERI-SimCenter/EE-UQ}{https://github.com/NHERI-SimCenter/EE-UQ}

\chapter{User Training}
\label{chap:training}
\input{Training.tex}

\chapter{Requirements}
\label{chap:requirements}
\input{Requirements.tex}

\chapter{Verification and Validation}
\label{chap:vnv}
The following section (will) contain examples that verify the functionality of the tool.

Examples
This section provides examples of using EE-UQ for uncertainty quantification of structural analysis models used in earthquake engineering. 
Results of each model is verified against results obtained using other tools.
Two-Dimensional Portal Frame subjected to Gravity and Earthquake Loading
In this example, a simple 2D portal frame model is used to verify the results of $EE-UQ$. 
The model is a linear elastic single-bay single-story model of a reinforced concrete portal frame (\autoref{fig:figure20}). 
The analysis of this model considers both gravity loading and lateral earthquake loading due to El Centro earthquake 
(Borrego Mountain 04/09/68 0230, El Centro ARRAY \#9, 270). 
The original model and ground motion used in this example were obtained from 
\href{http://opensees.berkeley.edu/wiki/index.php/OpenSees_Example_1b._Elastic_Portal_Frame}{example 1b} in OpenSees website, 
and were modified to scale the ground motion record from gravity units (g) to the model units (in/sec2). 
Files for this example are included with the release of the software and are available in the Examples folder in a subfolder called PortalFrame2D.


\begin{figure}[!htbp]
  \centering {
    \includegraphics[width=0.8\textwidth]
    {figs/Figure20.png} }
  \caption{Two-dimensional portal frame model subjected to gravity and earthquake loading}
  \label{fig:figure20}
\end{figure}

To introduce uncertainty in the model, both mass and young’s modulus are assumed to be normally distributed random variables with means and 
standard deviation values shown in Table 1. In this example, the model will be sampled with the Latin Hypercube sampling method using both 
EE-UQ and a Python script (PortalFrameSampling.py) and response statistics from both analyses will be compared.


\begin{table}[hbt!]                       
  \centering
\begin{adjustbox}{max width=\textwidth}            
  \begin{tabular}{lllll}                    
    \toprule          
      Uncertain Parameter & 	Distribution	 &  Mean  &  Standard Deviation \\ \hline
	Nodal Mass, m [kip]	 & Normal & 	5.18	 & 1.0 \\ \hline
	Young’s Modulus, E [ksi] & 	Normal	 & 4227	 & 500.0 \\ \hline
  \end{tabular}
\end{adjustbox}
  \caption{ Uncertain parameters defined in the portal frame model}             
  \label{tab:uncertainty}                 
\end{table}



Modeling uncertainty using EE-UQ can be done using the following steps:
\begin{enumerate}
\item	 Start EE-UQ, click on the simulation tab (SIM) in the left bar to open a building simulation model. Click on choose button in the input script row:

\begin{figure}[!htbp]
  \centering {
    \includegraphics[width=0.8\textwidth]
    {figs/Figure21.png} }
  \caption{Choose building model}
  \label{fig:figure21}
\end{figure}

\item	 Choose the model file Portal2D-UQ.tcl from PortalFrame2D example folder.
\begin{figure}[!htbp]
  \centering {
    \includegraphics[width=0.8\textwidth]
    {figs/Figure22.png} }
  \caption{Choose tcl file}
  \label{fig:figure22}
\end{figure}


\item	 In the list of Clines Nodes edit box, enter “1, 3”. This indicates to EE-UQ that nodes 1 and 3 are the nodes used to obtain EDP at different floor levels (i.e. base and first floor).
\begin{figure}[!htbp]
  \centering {
    \includegraphics[width=0.8\textwidth]
    {figs/Figure23.png} }
  \caption{Select nodes}
  \label{fig:figure23}
\end{figure}

\item Click on the event tab (EVT) in the left bar to open the earthquake event specification tab, select Multiple Existing for loading Type. Click on the add button to add an earthquake event. 
Then click on the choose button to select the event file.
\begin{figure}[!htbp]
  \centering {
    \includegraphics[width=0.8\textwidth]
    {figs/Figure24.png} }
  \caption{Work on EVT tab}
  \label{fig:figure24}
\end{figure}

\item Choose the event file (BM68elc.json) for El Centro earthquake provided in the portal frame 2D example folder.
\begin{figure}[!htbp]
  \centering {
    \includegraphics[width=0.8\textwidth]
    {figs/Figure25.png} }
  \caption{Choose event file}
  \label{fig:figure25}
\end{figure}

\item Now select the random variables tab (RVs) from the left bar, change the random variables types to normal 
and set the mean and standard deviation values of the floor mass and Young’s modulus. 
Notice that EE-UQ has automatically detected parameters defined in the OpenSees tcl file using the pset command and defined them as random variables.
\begin{figure}[!htbp]
  \centering {
    \includegraphics[width=0.8\textwidth]
    {figs/Figure26.png} }
  \caption{Work on RVs tab}
  \label{fig:figure26}
\end{figure}

\item Now click on run, set the analysis parameters, working directory and applications directory and click submit to run the analysis. 
If everything ran successfully the program will automatically open the results tab showing the summary of results (\autoref{fig:figure27}).
\begin{figure}[!htbp]
  \centering {
    \includegraphics[width=0.8\textwidth]
    {figs/Figure27.png} }
  \caption{Run}
  \label{fig:figure27}
\end{figure}

\end{enumerate}



Verification script
A verification script (Listing 1) for propagating the uncertainty was developed in Python and is included in the example folder. 
The script creates 1000 samples for both the Young’s modulus and mass values using Latin Hypercube sampling, 
then modifies the OpenSees model, runs it and stores the output. 
After all the model samples are processed, the script will compute and output the mean and standard deviation values of the peak floor acceleration and peak drift.

{\tiny
\begin{lstlisting}[caption=Python script for analyzing the portal frame model with uncertain parameters]
import numpy as np
import os
import shutil
import subprocess
from pyDOE import *
from scipy.stats.distributions import norm

#Setting number of samples
nSamples = 1000

#Creating latin hyper cube designs
design = lhs(2, samples=nSamples)

#Sampling Young's Modulus and Mass
ESamples = norm(loc=4227, scale=500.0).ppf(design[:,0])
mSamples = norm(loc=5.18, scale=1.0).ppf(design[:,1])

#Initializing output arrays
PFA = []
PID = []
#Reading OpenSees Model
with open ("Ex1b.Portal2D.EQ.tcl", "r") as portalFrameFile:
    portalFrameModel = portalFrameFile.read()

    #Looping through the samples and creating modified models
    for i in range(nSamples):
        sampleName = str(i+1)
        if(os.path.exists(sampleName) and os.path.isdir(sampleName)):
            shutil.rmtree(sampleName)

        os.mkdir(sampleName)
        shutil.copy('BM68elc.acc', sampleName)

        #Modifying the model using sample E and m values
        with open (sampleName + '/Ex1b.Portal2D.EQ.tcl' , "w+") as modifiedFile:
            modifiedModel = portalFrameModel.replace('pset floorMass 5.18', 'pset floorMass ' + str(mSamples[i]))
            modifiedModel = modifiedModel.replace('pset E 4227', 'pset E ' + str(ESamples[i]))
            modifiedFile.write(modifiedModel)

        #Running OpenSees
        subprocess.Popen("OpenSees Ex1b.Portal2D.EQ.tcl", shell=True, cwd=sampleName).wait()

        #Reading Peak Floor Acceleration
        with open (sampleName + '/PFA.out' , "r") as pfaFile:
            PFA.append(float(pfaFile.readlines()[2]))

        #Reading Peak Floor Acceleration
        with open (sampleName + '/PID.out' , "r") as pidFile:
            PID.append(float(pidFile.readlines()[2]))

        #Cleaning up
        shutil.rmtree(sampleName)

#Printing results
print 'Mean Peak Floor Acceleration: ', np.mean(PFA)
print 'Peak Floor Acceleration Std. Dev: ', np.std(PFA)

print 'Mean Peak Drift: ', np.mean(PID)
print 'Peak Drift Std. Dev.: ', np.std(PID)

\end{lstlisting}
}


Verification of results
In this section, the results produced for the portal frame by EE-UQ are verified against the results of running the same problem using the Python script. 
Running the uncertainty quantification problem on the local computer produces the results shown in \autoref{fig:figure28} 
Running the analysis using the sampling Python script produces the results shown in \autoref{fig:figure29}. 
Both results (Mean and standard deviation values of EDPs) are compared in Table 2 and are shown to be in good agreement.
\begin{figure}[!htbp]
  \centering {
    \includegraphics[width=0.8\textwidth]
    {figs/Figure28.png} }
  \caption{Outputs from EE-UQ}
  \label{fig:figure28}
\end{figure}


\begin{figure}[!htbp]
  \centering {
    \includegraphics[width=0.8\textwidth]
    {figs/Figure29.png} }
  \caption{Outputs from PortalFrameSamplying.py script}
  \label{fig:figure29}
\end{figure}




\begin{table}[hbt!]                 
  \centering
\begin{adjustbox}{max width=\textwidth}            
  \begin{tabular}{lllll}                    
    \toprule          
      Engineering Demand Parameter &	 & EE-UQ	& Python Script	 & Percent Difference [\%]  \\ \hline
    
	\multirow{2}{*}{Peak Floor Acceleration [in/$s^2$]} 
	 & Mean &	67.4377	& 67.5448	& 0.16 \\
      & Std. Dev.	& 12.6487	 & 12.5487	& 0.8 \\ \hline
      
      \multirow{2}{*}{Peak Story Drift [x10-3 in]} 
      & Mean &	1.3428 &	1.347 &	0.3 \\
      & Std. Dev.	& 0.2832 &	0.2955	& 4.1	 \\

      \bottomrule      
                            
  \end{tabular}
\end{adjustbox}
  \caption{Features (M=Mandatory, D=Desirable, O=Optional, P=Possible Future)}             
  \label{tab:edp}                 
\end{table}

\nocite{*}

% \appendix
% \chapter{More Monticello Candidates}

\pagestyle{plain}
{
  \renewcommand{\thispagestyle}[1]{}	
  \printbibliography           
}

\end{document}
