This is provided for the user to specify multiple existing SimCenter
Event files.  If more than one event is specified it is done to provide
the UQ engine with a discrete set of events to choose from.  It is not
done with the intention of specifying that one event follows another.
The panel presented initially to the user is as shown
in \autoref{fig:figure4}.

\begin{figure}[!htbp]
  \centering {
    \includegraphics[width=0.8\textwidth]
    {usage/figures/multipleExisting1.png} }
  \caption{\texttt{Multiple Existing} events selected as EVT loading type}
  \label{fig:figure4}
\end{figure}

To add a new event, the user presses the Add button.  This adds an event to the panel.  Pressing the button multiple times will keep adding events to the panel.  \autoref{fig:figure5} shows the state after the button has been pressed twice, and data entered for the ElCentro and Rinaldi Events.

\begin{figure}[!htbp]
  \centering {
    \includegraphics[width=0.8\textwidth]
    {usage/figures/multipleExisting2.png} }
  \caption{Adding new event under \texttt{Multiple Existing} loading type}
  \label{fig:figure5}
\end{figure}

The user can enter the full path manually to the file or use the
choose button, which brings up your typical file search screen.  By
default, a scaling factor of 1.0 is assigned to the event.  The user
can change this to another real value (AT PRESENT DO NOT USE INTEGER)
or the user has the option of defining this to be a random variable by
entering a name as shown for the second event.

Note that this variable name must not start with a number, or contain
any spaces or special characters, i.e. no -, +,..

The Remove button is pressed to remove events. To remove an event the
user must first select which events they wish to remove, done by
clicking in the small circle at the start of the event. Once the
events to remove have been selected, the user removes all these
selected evens by pressing the remove button.

If the user has multiple events to load, all the event files may first
be placed by the user into a seperate folder. If the user presses the
Load Directory, the user will be able to choose a directory and the
application will load all the event file (any file with a .json
suffix) into the widget by choosing the directory. Initially each
event will be given a load factor of 1.0.  Should the user include in
that directory a file named \texttt{Records.txt} the application will open that
file and load the events and assigned load factors from that
file. Each line in Recors.txt is considered ro represent a record, and
contains 2 comma seperated values: the first value being the event
file and the second value the event factor. An example \texttt{Records.txt} is
as shown below:

\begin{verbatim}
ElCentro.json,1.5
Rinaldi.json,2.0
\end{verbatim}

Random Variables: The user can, as mentioned, enter a string in the
factor field to specify that the factor is to be considered a random
variable. Subsequently in the UQ panel the user must provide
information on the random variables distribution. Also, if multiple
events are specified, the event itself will be treated as a random
variable, with each event being part of the discrete set of possible
events.
