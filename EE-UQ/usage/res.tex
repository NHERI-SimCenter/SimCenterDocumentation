When the user hits the Run button, and assuming the results are
successful. The results are presented here.  A successful run or
download of a job that ran successfully will result in 3 tabbed
widgets being displayed in this panel.  The first panel, shown in
\Cref{fig:results_summary} shows summary statistics: mean and
stdDev values or min-max values if discrete set, i.e. multiple events
for each of the EDP's specified in the EDP panel. The second panel,
shown in \Cref{fig:summary_information} shows the summary
information.

\begin{figure}[!htbp]
  \centering {
    \includegraphics[width=0.8\textwidth]
    {usage/figures/resultsSummary.png} }
  \caption{Results Summary}
  \label{fig:results_summary}
\end{figure}

\begin{figure}[!htbp]
  \centering {
    \includegraphics[width=0.8\textwidth]
    {usage/figures/resultsInformation.png} }
  \caption{\texttt{General} tab showing results summary information}
  \label{fig:summary_information}
\end{figure}

The third panel, shown in \Cref{fig:results_data} presents
graphically and in tabular form the results. By selecting different
columns with left and right mouse buttons in the table below the
graphic, the information in the graph is changed. Selecting the left
mouse button changes the Y axis, the right mouse changes the X
axis. If the same column is selected using both left and right keys,
the CDF and PDF is displayed. If last mouse press was with the left
button, the PDF and if right the CDF.
 
As for the columns. You will see a column for each random variable the
workflow came across. There may be more than you specified if the
applications want the UQ engine to consider their own variables in the
computation. The outputs at present are limited to:

\begin{figure}[!htbp]
  \centering {
    \includegraphics[width=0.8\textwidth]
    {usage/figures/resultsData.png} }
  \caption{Results presented graphically and in tabular form}
  \label{fig:results_data}
\end{figure}
