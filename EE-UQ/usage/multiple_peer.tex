This is provided for the user to specify multiple existing PEER
(\href{http://peer.berkeley.edu}{http://peer.berkeley.edu}) ground
motion files.  For PEER events the user is required to specify the
individual components for the EVENTS.  The Add/Remove buttons at the
top are to create and remove an event, as per 2.2.1.  For the PEER
events the user specifies components acting in the individual
degree-of-freedom directions.  The + and – add and remove components
with the remove removing all components selected.  Each component in a
PEER event can have their own scale factor, again a number or a random
variable.

\begin{figure}[!htbp]
  \centering {
    \includegraphics[width=0.8\textwidth]
    {usage/figures/multiplePEER.png} }
  \caption{\texttt{Multiple PEER} loading type}
  \label{fig:figure6}
\end{figure}

If the user has multiple events to load the user can again place att the PEER .AT2 files into a separate folder and select the Load Directory option. This will allow the user to select a directory. Once selected all .AT2 files in that directory will be loaded into the application. Similar to loading multiple SimCenter events, should the user provide a file \texttt{Records.txt} in that directory, the application will load all files in the list and set the appropriate load factor. An example Results.txt file for multiple Peer events is as shown below:

\begin{verbatim}
elCentro.AT2,1.5
Rinaldi228.AT2,2.0
Rinaldi318.AT2,2.0
\end{verbatim}

Random Variables: The user can, as mentioned, enter a string in the
factor field to specify that the factor is to be considered a random
variable. Subsequently in the UQ panel the user must provide
information on the random variables distribution. Also, if multiple
events are specified, the event itself will be treated as a random
variable, with each event being part of the discrete set of possible
events.
