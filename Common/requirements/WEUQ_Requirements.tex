\begin{longtable}{| p{.07\textwidth} | p{.65\textwidth} | p{.08\textwidth} | p{.08\textwidth} |  p{.08\textwidth} |}
                               \toprule
       \# & Description & Source & Priority & Version \\ \hline
1 & \textbf{Ability to determine response of Building Subject to Wind Loading including formal treatment of randomness and uncertainty uncertainty} & GC & M & 1.0  \\ \hline
1. & Simulations able to utilize HPC resources & GC & M & 1.0 \\ \hline
1. & Simulations able to utilize HPC resources & GC & M & 1.0 \\ \hline
 1. & Tool should incorporate data from www & GC & M & 1.0 \\ \hline
 1.3 & Tool available for download from web & GC & M & 1.0 \\ \hline
          2 & \textbf{Wind Loading Options } &  &  \\ \hline
2.1 & Utilize Extensive wind tunnel datasets in indutsry and academia for wide range of building shapes & GC & M & 2.0 \\ \hline
2.1.1 & Accomodate Range of Low Rise building shapes & SP & M & 2.0 \\ \hline
2.1.1.1 & Flat Shaped Roof - TPU dataset and user provided & SP & M & 2.0 \\ \hline
2.1.1.2 & Gable Shaped Roof - TPU and User provided & SP & M & \\ \hline
2.1.1.3 & Hipped Shaped Roof - TPU and User provided & SP & M & \\ \hline
2.1.2 & Accomodate Range of High Rise building shapes & SP & M & 2.0 \\ \hline
2.1.2.1 & Cuboid  - TPU dataset and user provided & SP & M & 2.0 \\ \hline
2.2 & Computational Fluid Dynamics tool for utilizing open source CFD software for practitioners & GC & M & 1.0 \\ \hline
2.2.1 & Simple CFD model generation and turbulance modeling & GC & M & 2.0 \\ \hline
2.2.2 & Uncoupled OpenFOAM CFD model with nonlinear FEM code for building response & SP & M & 1.0 \\ \hline
2.2.3 & Coupled OpenFOAM CFD model with nonlinear FEM code for building response & SP & M &  \\ \hline
2.3 & Quantification of Effects of Wind Bourne Debris & GC & D & \\ \hline
2.4 & Application to utilize GIS and online to account for wind speed given local terrain, topography and nearby buildings & GC & D & \\ \hline
2.5 & Ability to utilize synthetic wind loadings & SP & M & 1.0  \\ \hline
2.5.1 & per Wittig and Sinha & SP & D & 1.1  \\ \hline
3 & \textbf{Building Model Generation} & GC & 2.0 \\ \hline
3.1 & Ability to quickly create a simple nonlinear building model & GC & D & 1.1 \\ \hline
3.2  & Ability to define building and use Expert System to generate FE mesh & SP & &  \\ \hline
	3.2.1 & Expert system for Concrete Shear Walls & SP & M &  \\ \hline
	3.2.2 & Expert system for Moment Frames & SP & M &  \\ \hline
	3.2.3 & Expert system for  Braced Frames & SP & M &   \\ \hline
3.3 & Ability to define building and use Machine Learning applications to generate FE & GC &  &  \\ \hline
	3.3.1 & Machine Learning for Concrete Shear Walls & SP & M &  \\ \hline
	3.3.2 & Machine Learning for Moment Frames & SP & M &  \\ \hline
	3.3.3 & Machine Learning for Braced Frames & SP & M &   \\ \hline
	3.4 & Ability to specify connection details for member ends & SP & M & 2.2 \\ \hline
	3.5 & Ability to define a user-defined moment-rotation response representing the connection details & SP & D & 2.2 \\ \hline
	4 & \textbf{perform nonlinear Structural Analysis} &  &  \\ \hline
4.1 & Ability to use utilize existing nonlinear analysis software used in earthquake engineering & GC & M & 1.0 \\ \hline
4.1.1 & Utilize open source OpenSees software & SP & M & 1.0 \\ \hline
4.2.1 & Ability to provide own OpenSees Analysis script to OpenSees engine. & SP & D & 1.0 \\ \hline
4.3.1 & Ability to provide own Python script and use OpenSeesPy engine. & SP & O & 1.2 \\ \hline
4.2 & Ability to use alternative FEM engine & SP & M & 2.0 \\ \hline
	5 & \textbf{UQ - Method} &  GC &  \\ \hline
	5.1 & \textbf{UQ - Forward Propogation Methods} & SP  &  \\ \hline
	5.1.1 & Ability to use basic  Monte Carlo and LHS methods & SP & M & 1.0 \\ \hline
	5.1.2 & Ability to use Importance Sampling  & SP & M & 2.0 \\ \hline
	5.1.3 & Ability to use Gaussian Process Regression & SP & M & 2.0 \\ \hline
	5.1.4 & Ability to use Own External UQ Engine & SP & M &  \\ \hline
	5.2 & \textbf{UQ - Reliability Methods} & UF &  &  \\ \hline
	5.2.1 & Ability to use First Order Reliability method & UF & M &  \\ \hline
	5.2.2 & Ability to use Second Order Reliability method & UF & M & \\ \hline
	5.2.2 & Ability to use Surrogate Based Reliability & UF & M & \\ \hline
	5.2.3 & Ability to use Own External Application to generate Results & UF & M &  \\ \hline
	5.3 & \textbf{UQ - Sensitivity Methods} & UF &  &  \\ \hline
	5.3.1 & Ability to obtain Global Sensitivity Sobol's indices & UF & M &  \\ \hline
    6 & \textbf{UQ – Random Variables} &  &  \\ \hline
    6.1 & Ability to Define Variables of certain types: & GC &  &  \\ 
    6.1.1 &  Normal & SP & M  & 1.0 \\ \hline
    6.1.2 &  Lognormal & SP & M & 1.0 \\ \hline
    6.1.3 & Uniform & SP & M & 1.0  \\ \hline
    6.1.4 & Beta & SP & M & 1.0 \\ \hline
    6.1.5 & Weibull &  SP & M  & 1.0 \\ \hline
    6.1.6 & Gumbel &  SP & M & 1.0  \\ \hline
    6.2 & User defined Distribution & SP & M &  \\ \hline
    6.3 & Define Correlation Matrix & SP & M &  \\ \hline
     7 & Tool to allow user to load and save user inputs & SP & M & 1.0 \\ \hline
    8 & \textbf{Application Outputs} &  &  \\ \hline
    8.1 & Ability to see pressure distribution on building & GC & M &   \\ \hline
    8.2 & Ability to obtain basic building responses & SP & M &   \\ \hline
    8.3 & Ability to Process own Output Parameters & UF & M & 1.1  \\ \hline
    9 & \textbf{Documentation} &  &  \\ \hline
    9.1 & Documentation exists on tool usage & SP & M & 1.1  \\ \hline
    9.2 & Video Exists demonstrating usage & SP & M & 1.1  \\ \hline
    9.3 & Verification Examples Exist & SP & M & 1.1  \\ \hline
    10 & \textbf{Misc.} &  &  \\ \hline
    10.1 & Installer  which installs application and all needed software & UF & M &   \\ \hline
	\bottomrule 
\caption{Requirements for WE-UQ}
  \label{tab:featureRequirements}                 
\end{longtable}

Feature Requirements (M=Mandatory, D=Desirable, O=Optional, P=Possible Future)

