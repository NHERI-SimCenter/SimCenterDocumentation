The quoFEM is an opensource software tool that allows engineers to incorporate uncertainty quantification to natural hazards. It has been developed at the SimCenter, within the University of California, Berkeley. The SimCenter is part of the Natural Hazards Engineering Research Infrastructure (NHERI) program, funded by the National Science Foundation. 

The intended audience for this tool is researchers and practitioners interested in predicting the response of a system under uncertainty. This documentation summarizes the usability and capabilities of this software tool.  Major enhancements are expected to be included in future releases of quoFEM.

This open-source research application, the source code of which is
available at
the \href{https://github.com/NHERI-SimCenter/EE-UQ}{\texttt{\getsoftwarename{}}
Github page}, provides an application that can be used to predict the
response of a system under uncertainty. The application
is focused on quantifying the uncertainties in the predicted response.
 In this application, the user is required to
characterize the uncertainties in the input. The application will,
after utilizing the users selected sampling method, provide
information that characterizes the uncertainties in the computed
response measures. As the computations to make these determinations
can be prohibitively expensive to perform on a user's local computer,
the user has the option to perform the computations remotely on the
Stampede2 supercomputer. Stampede2 is located at the Texas Advanced
Computing Center (TACC) and made available to the user through NHERI
DesignSafe, the cyberinfrastructure provider for the distributed NSF
funded Natural Hazards in Engineering Research Infrastructure (NHERI)
facility.\\

Whether running locally or remotely, the computations are performed, in a workflow
application. The design of the \texttt{\getsoftwarename{}} application is such that researchers are able to modify the backend application to utilize their own application in the workflow
computations. This will ensure researchers are not limited to using
the default applications we provide and will be enthused to provide
their own applications for others to use. \\

This document covers Version \getsoftwareversion{} of the tool. Users are
encouraged to comment on what additional features and capabilities
they would like to see in this application. These requests and
feedback can be submitted through an anonymous \insertsurveylink{user
survey}; we greatly appreciate any input you have. If there are
features you want, chances are many of your colleagues also would
benefit from them. Users are encouraged to review
\Cref{chap:requirements} to see what features are planned for this
application.
