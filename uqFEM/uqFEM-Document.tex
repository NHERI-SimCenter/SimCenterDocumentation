\documentclass{simcenterdocumentation}
\usepackage[backend=biber]{biblatex}
%\usepackage{subfig}
\usepackage{subcaption}
\usepackage{multirow}
\usepackage{adjustbox}
\usepackage{cleveref}
\usepackage{longtable}

%% SETTING ENVIRONMENT FOR PYTHON CODE SNIPPETS %%%%%%%%%%%%%%%%%%%%%%%%%%%%%%% 
\usepackage[utf8]{inputenc}

\graphicspath{{../Common/}{.}} %Setting the graphicspath
\makeatletter % Search additional directories for inputs
\def\input@path{{../Common/}{.}}
%or: \def\input@path{{/path/to/folder/}{/path/to/other/folder/}}
\makeatother

%%%%%%%%%%%%%%%%%%%%%%%%%%%%%%%%%%%%%%%%%%%%%%%%%%%%%%%%%%%%%%%%%%%%%%%%%%%%%%% 

% To compile this file, run "latex/pdflatex codedoc", then "biber codedoc"
% (or "bibtex codedoc", if the output from latex asks for that instead),
% and then "latex/pdflatex codedoc" (without the quotes in each case).

% Double spacing, if you want it.  Do not use for the final copy. Can also specify
% draft as a document class option. This will generate double spacing and placeholders
% for title page and header images
%% \def\dsp{\def\baselinestretch{2.0}\large\normalsize}
%% \dsp

\bibliography{../Common/references}

\begin{document}

% Declarations for Front Matter
% Software title followed by optional second line
\title{quoFEM\\ \Large Quantification of Uncertainty \& Optimization in Finite Element Modeling}
% Use superscripts to indicate author affiliations
\author{Frank McKenna, Nikhil Padhye, Adam Zsarnoczay, \& Alexandros Taflanidis}
%\author{Moe Howard$^{1,2}$ Larry Fine$^1$ Curly Howard$^2$}
\institutions{NHERI SimCenter, UC Berkeley}
\softwarename{quoFEM}
\softwareversion{2.0}
\softwarepage{https://simcenter.designsafe-ci.org/research-tools/uqfem-application/}

%%% DON'T MESS WITH THESE SETTINGS %%%%%%%%%%%%%%%%%%%%%%%%%%%%%%%%
\hypersetup{pageanchor=false}
\maketitle
\copyrightpage
\acknowledgments

\hypersetup{pageanchor=true}
\begin{frontmatter}

\pagestyle{plain}
{
  \renewcommand{\thispagestyle}[1]{}
  \tableofcontents
  \clearpage
  \listoffigures
  \clearpage
  \listoftables
}

\end{frontmatter}
\pagestyle{somewhatsimple}
%%%%%%%%%%%%%%%%%%%%%%%%%%%%%%%%%%%%%%%%%%%%%%%%%%%%%%%%%%%%%%%%%%%
% Create separate tex files for each chapter and provide them as inputs

\chapter{About}
\label{chap:about}
The quoFEM is an opensource software tool that allows engineers to incorporate uncertainty quantification to natural hazards. It has been developed at the SimCenter, within the University of California, Berkeley. The SimCenter is part of the Natural Hazards Engineering Research Infrastructure (NHERI) program, funded by the National Science Foundation. 

The intended audience for this tool is researchers and practitioners interested in predicting the response of a system under uncertainty. This documentation summarizes the usability and capabilities of this software tool.  Major enhancements are expected to be included in future releases of quoFEM.

This open-source research application, the source code of which is
available at
the \href{https://github.com/NHERI-SimCenter/EE-UQ}{\texttt{\getsoftwarename{}}
Github page}, provides an application that can be used to predict the
response of a system under uncertainty. The application
is focused on quantifying the uncertainties in the predicted response.
 In this application, the user is required to
characterize the uncertainties in the input. The application will,
after utilizing the users selected sampling method, provide
information that characterizes the uncertainties in the computed
response measures. As the computations to make these determinations
can be prohibitively expensive to perform on a user's local computer,
the user has the option to perform the computations remotely on the
Stampede2 supercomputer. Stampede2 is located at the Texas Advanced
Computing Center (TACC) and made available to the user through NHERI
DesignSafe, the cyberinfrastructure provider for the distributed NSF
funded Natural Hazards in Engineering Research Infrastructure (NHERI)
facility.\\

Whether running locally or remotely, the computations are performed, in a workflow
application. The design of the \texttt{\getsoftwarename{}} application is such that researchers are able to modify the backend application to utilize their own application in the workflow
computations. This will ensure researchers are not limited to using
the default applications we provide and will be enthused to provide
their own applications for others to use. \\

This document covers Version \getsoftwareversion{} of the tool. Users are
encouraged to comment on what additional features and capabilities
they would like to see in this application. These requests and
feedback can be submitted through an anonymous \insertsurveylink{user
survey}; we greatly appreciate any input you have. If there are
features you want, chances are many of your colleagues also would
benefit from them. Users are encouraged to review
\Cref{chap:requirements} to see what features are planned for this
application.


\chapter{Installation Instructions}
\label{chap:installation}
\input{installation/installation_quoFEM.tex}

\chapter{Usage}
\label{chap:usage}
\section{User Interface}
The user interface (UI), as shown in \Cref{fig:generic_ui}, is where the analysis
is configured and managed. Here, the user is able to provide the necessary
parameters to create the simulation, start the simulation both locally and
remotely, and view the simulation results. The interface contains several
separate areas:

\begin{figure}[!htbp]
  \centering {
    \includegraphics[width=0.95\textwidth]
    {examples/fig_quofem/fw.png} }
  \caption{The User Interface (UI)}
  \label{fig:generic_ui}
\end{figure}

\begin{enumerate}
\item Input Panel Selection: This area on the left side provides the
  user with a selection of items to choose from:
\begin{enumerate}
  \item UQ: the input panel for specifying everything related to Uncertainty Quantification analysis.
  \item FEM: the input panel for Finite Element model specification.
  \item RV: the input panel for the specification of random input for the problem at hand.
  \item QoI: the input panel for specifying the desired output to extract from each Finite Element simulation.
  \item RES: results tab for outputting all tables, figures, results, graphs, from the quoFEM application.  
\end{enumerate}

Selecting any of these will change the input panel presented.

\item Push Buttons: This is the area near the bottom of the UI in
  which 4 buttons are presented to the user:

\begin{enumerate}
\item RUN – Run the simulation locally on the user’s desktop machine.
\item RUN at DesignSafe – Process the information, and send to
  DesignSafe. The simulation will be run there on a supercomputer, and results
  will be stored in the user's DesignSafe jobs folder.
\item GET from DesignSafe – Obtain the list of
  jobs for the user from DesignSafe and select a job to download from that list.
\item Exit: Exit the application.
\end{enumerate}

The first 3 of the above buttons and their use are discussed in more detail in \Cref{sec:push_buttons}.

\item Login Button: The Login Button is at the top right of the UI. Before the user can launch any jobs on DesignSafe, they must
  first login to DesignSafe using their DesignSafe login and
  password. Pressing the login button will open up the login window
  for users to enter this information. Users can register for an
  account on
  the \href{https://www.designsafe-ci.org/account/register/}{DesignSafe
  webpage}.

\item Message Area: While the application is running, error and status messages will be displayed here, in the top center of the user interface.

\end{enumerate}


\section{UQ: Uncertainty Quantification}
\label{sec:uq}
Throughout the input specification the user is defining variables. As
described in the above sections many of these variables can be
specified by the user to be random variables. The UQ panel is where the user specifies the distribution of these random variables. Besides the properties of random variables, the sampling method and the number of requested samples shall also be defined by the user. The panel is split, as shown
in \Cref{fig:uq_panel}, into two frames:

\begin{enumerate}
\item Sampling Methods 
\item Random Variables
\end{enumerate}

\begin{figure}[!htbp]
  \centering {
    \includegraphics[width=0.8\textwidth]
    {usage/figures/uq1.png} }
  \caption{Uncertainty Quantification input panel}
  \label{fig:uq_panel}
\end{figure}

\subsection{Sampling Methods}
In the \href{https://dakota.sandia.gov//sites/default/files/docs/6.9/html-ref/method-sampling.html}{sampling methods} the user selects the sampling 
method to use from the dropdown menu. Currently there are two options available: 
Monte Carlo and Latin Hypercube Sampling (LHS). Depending on the option selected, the user must specifies the number of samples and the seed. The number of samples specifies the number of simulations to be performed. Providing a random seed allows the user to reproduce the same set of samples from the random variables multiple times.

\subsection{Random Variables}
The Random Variable panel allows the user to characterize the random
variables. Each random variable has a name and a distribution. The
distribution is selected from the drop-down menu. By changing the
distribution type, the parameters required to define the distribution
change. The following distributions are available (clicking on a link will take you to the Dakota manual that provides theoretical background and explains the requested parameters for each distribution):
\begin{enumerate}
\item \href{https://dakota.sandia.gov//sites/default/files/docs/6.9/html-ref/variables-normal_uncertain.html}{Normal}
\item \href{https://dakota.sandia.gov//sites/default/files/docs/6.9/html-ref/variables-lognormal_uncertain.html}{Lognormal}
\item \href{https://dakota.sandia.gov//sites/default/files/docs/6.9/html-ref/variables-beta_uncertain.html}{Beta}
\item \href{https://dakota.sandia.gov//sites/default/files/docs/6.9/html-ref/variables-uniform_uncertain.html}{Uniform}
\item \href{https://dakota.sandia.gov//sites/default/files/docs/6.9/html-ref/variables-weibull_uncertain.html}{Weibull}
\item \href{https://dakota.sandia.gov//sites/default/files/docs/6.9/html-ref/variables-gumbel_uncertain.html}{Gumbel}
\end{enumerate} 

As with other panels, the random variables can be added or
removed. Care must be taken by the user in ensuring that if the user
removes random variables from this panel that they also remove them
from the other input widgets. Failing to do so may result in the
program failing to complete.


\section{FEM: Finite Element Method}
\label{sec:fem}
The FEM panel will present users with a selection of FEM
applications. Currently, there is two application
available, OpenSees and FEAPpv. More FE platforms will be added 
 in future versions to allow users to provide their own
simulation application.  

\begin{figure}[!htbp]
  \centering {
    \includegraphics[width=0.8\textwidth]
    {examples/fig_quofem/fem.png} }
  \caption{Options for FEM file specification with OpenSees}
  \label{fig:fem}
\end{figure}

\begin{figure}[!htbp]
  \centering {
    \includegraphics[width=0.8\textwidth]
    {examples/fig_quofem/fem2.png} }
  \caption{Options for FEM file specification with FEAPpv}
  \label{fig:fem2}
\end{figure}

For the OpenSees and FEAPpv applications, the user is required to specify the FEM input file, as shown in \Cref{fig:fem} and \Cref{fig:fem2}, respectively. A user-provided post-processing script for each FEM application needs to be selected appropriately. 

\section{RV: Random Variables}
\label{sec:rv_quofem}
The RV panel allows the user to specify the probabilistic distribution for the random problem at hand. The following probabilistic distributions for the random variables are currently supported: 
\begin{enumerate}
\item Gaussian
\item Lognormal
\item Beta
\item Uniform
\item Weibull
\item Gumbell
\end{enumerate}

Each distribution has different parameters, and the user needs to select accordingly the parameters for the distribution selected for each random variable. Once the user selects the distribution of the random variable, the
corresponding input boxes for the parameters will show. 

\Cref{fig:rv} shows the panel for a problem with four Random Variables with all random input following Gaussian distributions. 

\begin{figure}[!htbp]
  \centering {
    \includegraphics[width=0.8\textwidth]
    {examples/fig_quofem/rv.png} }
  \caption{Random Variable specification}
  \label{fig:rv}
\end{figure}




\section{QoI: Output}
\label{sec:qoi_quofem}
The QoI panel provide the user the capability to specify the output of the FEM results. The user needs to specify the name tag associated with the random output of the FEM simulation to which uncertainty results need to be produced. 

\section{RES: Results}
\label{sec:res_quofem}
\input{usage/res_quofem}


\chapter{Verification and Validation}
\label{chap:vnv}
This chapter provides examples of using the \texttt{\getsoftwarename{}} application for uncertainty
quantification of Finite Element models. Results of each model are verified against results
obtained using other tools. 
\\

\section{Two-Dimensional Truss Structure}
Consider the problem of uncertainty quantification in a two-dimensional truss structure, as shown in figure \Cref{fig:truss}. 

The structure has uncertain properties that all follow normal distribution such that:
\begin{enumerate}
\item Elastic moduli: $\bar{E}=205kN/mm^2$, and $\sigma_{E}=15 kN/mm^2$ 
\item Load: $\bar{P}=25kN$, $\sigma_{P}=2.5kN$
\item Cross sectional area for the upper three bars: $A_u=500mm^2$, and $\sigma_{A_u}=25mm^2$
\item  Cross sectional area for the other six bars: $A_l=250mm^2$, and $\sigma_{A_l}=10mm^2$
\end{enumerate}

Table \Cref{tab:uncertainty} summarizes the uncertain input mentioned above. 

\begin{figure}[!htbp]
  \centering {
    \includegraphics[width=0.8\textwidth]
    {examples/fig_quofem/truss.png} }
  \caption{Two-dimensional truss model used in Verification.}
  \label{fig:truss}
\end{figure}

\begin{table}[hbt!]                       
  \centering
\begin{adjustbox}{max width=\textwidth}            
  \begin{tabular}{lllll}                    
    \toprule          
      Uncertain Parameter & 	Distribution	 &  Mean  &  Standard Deviation \\ \hline
	Elastic Moduli $[kN/mm^2]$	 & Normal & 	205	 & 15 \\ \hline
	Load $[kN]$ & 	Normal	 & 25	 & 2.5 \\ \hline
  Cross section area (upper) $[mm^2]$ & Normal &   500  & 25 \\ \hline
  Cross section area (lower) $[mm^2]$ & Normal &   250  & 10 \\ \hline
  \end{tabular}
\end{adjustbox}
  \caption{Uncertain parameters defined in the portal frame model}             
  \label{tab:uncertainty}                 
\end{table}

We assume that the random variables are independent (i.e., zero covariance), and aim to estimate the mean and standard deviation of the vertical displacement at point F, with $95\%$ confidence intervals for both statistics. In this example, Dakota was used in conjunction with OpenSees. The setup example problem, populated entries, and sample results are show in the following figures.


Figures \Cref{fig:basic}, \Cref{fig:correlation}, \Cref{fig:cdfp}, \Cref{fig:cdfoutput}, and \Cref{fig:psamples} show the results corresponding to the forward problem, including sampling and surrogate-based. 

\begin{figure}[!htbp]
  \centering {
    \includegraphics[width=0.8\textwidth]
    {examples/fig_quofem/basic_summary.png} }
  \caption{Basic summary statistics for the specified output. }
  \label{fig:basic}
\end{figure}

\begin{figure}[!htbp]
  \centering {
    \includegraphics[width=0.8\textwidth]
    {examples/fig_quofem/correlation.png} }
  \caption{Input panel for the correlation matrix for the random variables of the truss problem. }
  \label{fig:correlation}
\end{figure}

\begin{figure}[!htbp]
  \centering {
    \includegraphics[width=0.8\textwidth]
    {examples/fig_quofem/p_cdf.png} }
  \caption{Samples of the random variables P along with the corresponding Cumulative Distribution Function for the truss problem. }
  \label{fig:cdfp}
\end{figure}

\begin{figure}[!htbp]
  \centering {
    \includegraphics[width=0.8\textwidth]
    {examples/fig_quofem/qoi_cdf.png} }
  \caption{Samples of the output variable along with the corresponding Cumulative Distribution Function for the 2D truss problem. }
  \label{fig:cdfoutput}
\end{figure}

\begin{figure}[!htbp]
  \centering {
    \includegraphics[width=0.8\textwidth]
    {examples/fig_quofem/run_plot.png} }
  \caption{Scatter plot of samples of the random variable P. }
  \label{fig:psamples}
\end{figure}





\chapter{Source Code}
\label{chap:SourceCode}
This source code for the tool is released under the 2-clause BSD
License, commonly called the FreeBSD license.  It is available for
download from the
tool's \href{https://github.com/NHERI-SimCenter/\getsoftwarename{}}{GitHub
repository}

\chapter{User Training}
\label{chap:training}
User Training consists of an online video available from the tool
webpage that demonstrates tool use. The tool will be presented in user
workshops hosted by the SimCenter.


\chapter{Requirements}
\label{chap:requirements}
This chapter outlines the general features of the \texttt{\getsoftwarename{}} application. We show when the features were introduced and what features and when you can expect to see in the future. This provides a roadmap of where this application has come from and where it is headed. The future features are highly dependent on user feedback. You are highly encouraged to contact us to discuss any new features you would like to see in the application.\\

\Cref{tab:schedule} shows the scheduled release dates for this tool and includes the list of features provided in that release or what you can expect to see in future releases. The individual feature requirements are outlined in \Cref{tab:featureRequirements}. If you would like some additional features added, please contact us.
Additionally, we would appreciate any feedback on this tool. An anonymous
user survey is available \insertsurveylink{here}. \\

\begin{table}[hbt!]                    
  \centering
\begin{adjustbox}{max width=\textwidth}            
  \begin{tabular}{lll}                    
    \toprule          
      Version & 	Release	 & Requirements \\  \hline
      2.0	 & October 2019 & 1.5, 1.7, 1.8, 2.1, 5.1, 5.2\\  \hline
  \end{tabular}
\end{adjustbox}
  \caption{Schedule of Release}             
  \label{tab:schedule}                 
\end{table}


\newpage
\begin{longtable}{| p{.05\textwidth} | p{.75\textwidth} | p{.08\textwidth} | p{.08\textwidth} |}
    \toprule
      \# & Description & Priority & Version \\ \hline
      1 & \textbf{Forward Uncertainty Propagation} &  &  \\ 
	1.1 & Input uncertainty characterization & M & 1.1 \\ \hline
	1.2 & PDF Approximation & M & 1.1 \\ \hline
	1.3 &  Descriptive output statistics & M & 1.1 \\ \hline
	1.4 &  Basic Monte Carlo Sampling  & M & 1.1 \\ \hline	
	1.5 &  Importance Sampling for rare events  & M & 2.0 \\ \hline	
	1.6 &  Cross-Entropy sampling  & M &  \\ \hline
	1.7 &  Forward Propagation, GPR Surrogate  & M & 2.0 \\ \hline
	1.8 &  Forward Propagation, PCE Surrogate  & M & 2.0 \\ \hline
	1.9 &  Multi-fidelity sampling  & M &  \\ \hline
	1.10 &  Spatial/temporal stochastic models  & M &  \\ \hline
	2 & \textbf{Sensitivity Analysis} &  &  \\ \hline
	2.1 & Global sensitivity Sobol's indices & M & 2.0  \\ \hline
	3 & \textbf{System Identification and Bayesian Inference} &  &  \\ \hline
	3.1 & Parameter estimation & M &  \\ \hline
	3.2 & Basic Bayesian parameter updating & M &  \\ \hline
	3.3 & Advanced MCMC-based Bayesian updating & M & \\ \hline
	3.4 & Advanced Surrogate-based Bayesian updating & M &  \\ \hline
	3.5 & Model class selection & M &  \\ \hline
	3.6 & Sequential Bayesian updating & M &  \\ \hline
	4 & \textbf{Optimization under Uncertainty} &  &  \\ \hline
	4.1 & Reliability-Based Design Optimization & M &  \\ \hline
	4.2 & Single-objective optimization under uncertainty & M &  \\ \hline
	4.3 & Multi-objective optimization under uncertainty & M &  \\ \hline
	5 & \textbf{Reliability Analysis} &  &  \\ \hline
	5.1 & First/Second Order Reliability Methods & M & 2.0 \\ \hline
	5.2 & Surrogate-based reliability & M & 2.0 \\ \hline
	\bottomrule
\caption{Feature Requirements (M=Mandatory, D=Desirable, O=Optional, P=Possible Future)}             
  \label{tab:featureRequirements}                 
\end{longtable}


\chapter{Troubleshooting}
\label{chap:troubleshooting}

\section{Problems Starting the Application}
\label{sec:startingProblems}
On Windows operating systems, if you receive an error when starting the application with the message that MSVCP140.dll is missing as shown in \Cref{fig:Missing_CRT_Error}, it is caused by a missing Visual C/C++ runtime library. You can fix this error by running the installer for the Visual C/C++ redistributable package (vc\_redist.x64.exe) which is included with the application.

\begin{figure}[!htbp]
  \centering {
  \softwareSwitch{PBE}{
  \includegraphics[width=0.8\textwidth]{troubleshooting/figures/PBE-MissingCRT.png}
  }{
  \includegraphics[width=0.8\textwidth]{troubleshooting/figures/EE-UQ-MissingCRT.png} 
  }
  }    
  \caption{Error message for missing Visual C/C++ runtime library}
  \label{fig:Missing_CRT_Error}
\end{figure}




\section{Problems Running Simulations}
\label{sec:simulationProblems}
The \texttt{\getsoftwarename{}} can be a complicated tool and it will not always run. Causes of failure include incorrect set up, non-functioning or poorly functioning websites, and of course user error. To discover the errors it is useful to understand how the UI and the backend work when the user submits to run a job. A number of things occur when the Submit button is clicked: 

\begin{enumerate}
\item The UI creates a folder in the working dir location specified called tmp.SimCenter and in that folder creates another folder called templatedir.
\item The UI then iterates through all the widgets chosen and these widgets place all needed files for the computation into the templatedir directory.
\item A python script is run in this templatedir directory that creates the input file for the UQ Engine. For example, using Dakota the input file dakota.in is created and placed in tmp.SimCenter folder.
\item The UQ engine is then started and runs using the dakota.in input file.
\item As the UQ engine runs, it creates folders in tmp.SimCenter, one folder for each deterministic run.
\item When completed the UQ engine leaves the results files in the tmp.SimCenter folder.
\item The results files are then processed by the UI and presented to the user in the RES tab.
\end{enumerate}

The following is a list of things that we have observed to go wrong when the UI informs the user of a failure and steps the user can take to fix the problem:  

\begin{enumerate}
\item \textbf{Could not create working dir}: The user does not have permission to create the tmp.SimCenter folder in working dir location. Change the Working Dir location in the window that pops up.
\item \textbf{No Script File}: The user has changed the Applications dir location, or the applications folder that accompanies the installation has been modified. Either set the correct dir location or re-install the tool.
\item \textbf{ERROR: Dakota failed to finish}: This can occur for a number of reasons. Go to the tmp.SimCenter folder and have a look for the dakota.err file.
\begin{enumerate}
\item \textbf{No dakota.err file and no dakota.in file}: the python script in templatdir failed to create the necessary files. Have a look at the workflow log file in templatedir folder to see what the error is as it could indicate an error in your input.
\item \textbf{No dakota.err and dakota.in exists}: Dakota failed to run. Check install of Dakota.
\item \textbf{dakota.err file exists}: Open the file and see what the error is.  For example if it says \textbf{Error: at least one variable must be specified.} This means no random variables have been specified. So whether you have only one  deterministic event or you have not specified any random variables in the EDP.
\item \textbf{dakota.err file exists but is empty}: This means that Dakota ran but there was a problem with the simulation. Go to one of the workdir locations. There is a file workflow driver that can be run. Run it and see what the errors are.
\item \textbf{You ran at DesignSafe and no dakota.out files come back}: Go to your data depot older at DesignSafe using the browser. Go to archive/jobs and use the job number shown in table that pops up when you ask to get the job from DesignSafe. look at the .err file in that directory for a clues to as what went wrong.
\item \textbf{No results and you used the Site Response to create the event}. You must run a simulated event in the Site Response Widget before you can submit a job to run.
\end{enumerate}
\end{enumerate}

If still having trouble, you can always join the \texttt{\getsoftwarename{}} slack channel and look for similar issues or post a new one.








\nocite{*}

\pagestyle{plain}
{
  \renewcommand{\thispagestyle}[1]{}	
  \printbibliography           
}

\end{document}
