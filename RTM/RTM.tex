\documentclass{simcenterdocumentation}
\usepackage[backend=biber]{biblatex}
%\usepackage{subfig}
\usepackage{subcaption}
\usepackage{multirow}
\usepackage{adjustbox}
\usepackage{cleveref}
\usepackage{longtable}

%% SETTING ENVIRONMENT FOR PYTHON CODE SNIPPETS %%%%%%%%%%%%%%%%%%%%%%%%%%%%%%% 
\usepackage[utf8]{inputenc}

\graphicspath{{../Common/}{.}} %Setting the graphicspath
\makeatletter % Search additional directories for inputs
\def\input@path{{../Common/}{.}}
%or: \def\input@path{{/path/to/folder/}{/path/to/other/folder/}}
\makeatother

%% Add unicode support for special characters
%%\usepackage[utf8x]{inputenc}

% To compile this file, run "latex/pdflatex codedoc", then "biber codedoc"b
% (or "bibtex codedoc", if the output from latex asks for that instead),
% and then "latex/pdflatex codedoc" (without the quotes in each case).

% Double spacing, if you want it.  Do not use for the final copy. Can also specify
% draft as a document class option. This will generate double spacing and placeholders
% for title page and header images
%% \def\dsp{\def\baselinestretch{2.0}\large\normalsize}
%% \dsp

\bibliography{../Common/references}

\begin{document}
% Declarations for Front Matter
% Software title followed by optional second line
\title{Requirements Traceability Matrix (RTM)}
% Use superscripts to indicate author affiliations
\author{Frank McKenna}
\institutions{NHERI SimCenter, UC Berkeley}
\softwarename{RTM}
\softwareversion{1.2}
\softwarepage{https://simcenter.designsafe-ci.org/research-tools/pbe-application/}

%%% DON'T MESS WITH THESE SETTINGS %%%%%%%%%%%%%%%%%%%%%%%%%%%%%%%%
\hypersetup{pageanchor=false}
\maketitle
%\copyrightpage
%\acknowledgments

\hypersetup{pageanchor=true}
\begin{frontmatter}

\pagestyle{plain}
{
  \renewcommand{\thispagestyle}[1]{}
  \tableofcontents
  \clearpage
%  \listoffigures
% \clearpage
  \listoftables
}

\end{frontmatter}
\pagestyle{somewhatsimple}
%%%%%%%%%%%%%%%%%%%%%%%%%%%%%%%%%%%%%%%%%%%%%%%%%%%%%%%%%%%%%%%%%%%
% Create separate tex files for each chapter and provide them as inputs
 
\chapter{Introduction}
\label{chap:about}

The RTM are tables linking requirements with project deliverables.  The requirements for the SimCenter have been obtained from a number of sources:
\begin{enumerate}
\item GC: From the vision documents referenced in the solicitation [2, 3, 5, 6] that outline the grand challenges for wind and earthquake hazards. These documents all present a list of research and educational advances needed that can contribute knowledge and innovation to prevent natural hazards from becoming societal disasters. The advances needed were identified through specially formed committees and workshops comprising researchers and practicing engineers. They identified both the grand challenges faced and also identified what was needed to address these challenges. The software needs identified in these reports that are applicable to research in natural hazards as permitted under the NSF NHERI program were identified. Those tasks that the NHERI SimCenter identified as pertaining to NHERI related research are considered here.
\item SP: From the senior personnel on the SimCenter project. The vision documents outline general needs without going into the specifics. The senior personnel on the project from these general needs identified specific requirements that could be provided that would provided a basis to allow research 
\item UF: SimCenter workshops, boot camps and direct user feedback. As the SImCenter develops and releases tools feedback from researchers using these tools is obtained at the tool training workshops, programmer boot-camps,  and through one-on-one discussion, email and online user feedback surveys. 
\end{enumerate}  

The software requirements are many. For ease of presentation they are broken into two groups:
\begin{enumerate}
\item Regional Scale - software to allow researchers and to examine the resilience of a community to natural hazard events.
\item Building Scale - software to allow researchers to improve on methods related to response assessment and performance based design of individual buildings subject to effect of a natural hazard.
\end{enumerate}


\chapter{Requirements}
\label{chap:requirements}

\section{Regional Scale}
\begin{longtable}{| p{.08\textwidth} | p{.58\textwidth} | p{.08\textwidth} | p{.08\textwidth} | p{.08\textwidth} | p{.08\textwidth} |}
    \toprule
\# & Description & Source & WBS & Need & Version \\ \hline
1 & Ability to perform regional simulation allowing communities to evaluate resilience and perform what-if types of analysis for natural hazard events & GC & M & 1.3.10 rWhale & 1.0 \\ \hline
1.1 & Ability to perform physics based simulation that integrates earth sciences, engineering, and social science for Earthquakes & GC 2.12 &  1.3.10 rWhale &  M & 1.0  \\ \hline
1.2 & Ability to perform physics based simulation that integrates earth sciences, engineering, and social science allowing communities to evaluate resilience and perform what-if types of analysis for Earthquakes & GC 2.12 &  1.3.10 rWhale &  M & 1.0  \\ \hline
1.3 & Hurricane & GC & 1.3.10 rWhale & M & \\ \hline
1.4 & Storm Surge & GC & 1.3.10 rWhale & M & \\ \hline
2 & Ability to incorporate damage to lifelines in determination of community resilience & GC & 1.3.10 rWhale & M & 1.0 \\ \hline
3 &  Ability to use GIS based loss estimation so communities can visualize hazard impacts & GC-2.7 & 1.3.10 RDT & M & \\ \hline
4 &  Ability to utilize HPC resources in regional simulations & GC-2.7 & 1.3.10 RDT & M & 1.0 \\ \hline
6 &  Ability to Utilize existing open-source software for faster deployment & GC 2.7 & 1.3.10 RDT & M & 1.0 \\ \hline
5 &  Provide open-source software for developers to test new data and algorithms & GC 2.7 & 1.3.10 RDT & M & 1.0  \\ \hline
7 &  Ability to perform validation studies to calibrate accuracy of models & GC 2.7 & 1.3.10 rWhale & M & 1.0   \\ \hline
8 &  Ability to incorporate improved damage and fragility models for buildings and lifelines & GC 2.7 & 1.3.10 rWhale pelicun  & M & 1.1 \\ \hline
9 &  Ability to incorporate improved indirect economic loss estimation models & GC 2.7 & 1.3.10 rWhale pelicun & M & \\ \hline
10 & Promote 'living' community risk models utilizing local inventory data from various scources & GC 2.7 & 1.3.0 rWhale & M & \\ \hline
10.1 & Developing and sharing standardized definitions, measurement protocols and strategies for data collection & GC 2.11 & 1.3.3 & M & \\ \hline
10.2 & Devloping tools that utilize GIS information and online images, e.g. google maps, for data collection & GC & 1.3.4 BRAILS & M 1.0 \\ \hline
10.2.1 & Predicting if building is a soft-story building for earthquake simulations & UF & 1.3.4 BRAILS & M & 1.0 \\ \hline
10.2.2 & Predicting roof shape of building for hurricane wind simulation & SP & 1.3.4 BRAILS & M & 1.0 \\ \hline
10.2.3 & Predicting level first floor of occupancy for hurricane storm surge simulation & SP & 1.3.4 BRAILS & M &  \\ \hline
10.2.4 & Providing instruction on gathering information from www for purposes of these regional simulations & UF & 1.2.3 Summer Bootcamp & M & 2019 \\ \hline
10.3 & Developing, sharing, and archiving datasets for analyzing and modeling resilience and vulnerability over time & GC 2.11 & 1.3.3 & M & \\ \hline
11 & Support researchers investigating new disaster events & GC 2.11 & & M & \\ \hline
12 & Ability to include multi-scale nonlinear models & GC 2.12 & 1.3.10 rWhale & M & 1.0 \\ \hline
13 & Ability to include a formal treatment of uncertainty and randomness & GC 2.12 & 1.3.10 rWhale & M & 1.0 \\ \hline
14 & Identify knowledge gaps and Promote NSF generated knowledge through regional demonstration projects that generate linkages to operational entities and decision makers & GC 2.18 & M & 1.4.2 &  \\ \hline
14.1 & Earthquake & SP & 1.4.2 Earthquake Testbeds & M & 1.0 \\ \hline
14.1.1 & Bay Area Testbed & SP & 1.4.2 Testbeds & M & 1.0 \\ \hline
14.1.2 & Anchorage Testbed & SP & 1.4.2 Testbeds & M & 1.0 \\ \hline
14.2 & Tsunami & SP & 1.4.2 Testbeds & M & \\ \hline
14.3 & Hurricane Wind & SP & 1.4.2 Testbeds & M & 2.0 \\ \hline
14.4 & Hurricane Wind and Storm Surge & SP & 1.4.2 Testbeds & M & \\ \hline
14.5 & Earthquake and Liefelines & SP & 1.4.2 Testbeds & M & \\ \hline

\caption{Requirements for Regional Simulations aiding Community Resilience}
  \label{tab:featureRequirements}                 
\end{longtable}

Feature Requirements (M=Mandatory, D=Desirable, O=Optional, P=Possible Future)





\section{Building Scale}

For the building scale simulations, the requirements are broken down by SimCenter application. There are a number of applications under development for each of the hazards. Many of the requirements related to UQ and analysis options are repeated amongst the different applications under the assumption that if it is beneficial to engineers dealing with one hazard, they will be beneficial to engineers dealing with other hazards.

\subsection{Response of Building to Wind Hazard}
The following are the requirements for response of single structure to wind.. The requirements are being met by the WE-UQ application. All items are related to work in WBS 1.3.7.

 \begin{longtable}{| p{.07\textwidth} | p{.65\textwidth} | p{.08\textwidth} | p{.08\textwidth} |  p{.08\textwidth} |}
                               \toprule
       \# & Description & Source & Priority & Version \\ \hline
1 & \textbf{Ability to determine response of Building Subject to Wind Loading including formal treatment of randomness and uncertainty uncertainty} & GC & M & 1.0  \\ \hline
1. & Simulations able to utilize HPC resources & GC & M & 1.0 \\ \hline
1. & Simulations able to utilize HPC resources & GC & M & 1.0 \\ \hline
 1. & Tool should incorporate data from www & GC & M & 1.0 \\ \hline
 1.3 & Tool available for download from web & GC & M & 1.0 \\ \hline
          2 & \textbf{Wind Loading Options } &  &  \\ \hline
2.1 & Utilize Extensive wind tunnel datasets in indutsry and academia for wide range of building shapes & GC & M & 2.0 \\ \hline
2.1.1 & Accomodate Range of Low Rise building shapes & SP & M & 2.0 \\ \hline
2.1.1.1 & Flat Shaped Roof - TPU dataset and user provided & SP & M & 2.0 \\ \hline
2.1.1.2 & Gable Shaped Roof - TPU and User provided & SP & M & \\ \hline
2.1.1.3 & Hipped Shaped Roof - TPU and User provided & SP & M & \\ \hline
2.1.2 & Accomodate Range of High Rise building shapes & SP & M & 2.0 \\ \hline
2.1.2.1 & Cuboid  - TPU dataset and user provided & SP & M & 2.0 \\ \hline
2.2 & Computational Fluid Dynamics tool for utilizing open source CFD software for practitioners & GC & M & 1.0 \\ \hline
2.2.1 & Simple CFD model generation and turbulance modeling & GC & M & 2.0 \\ \hline
2.2.2 & Uncoupled OpenFOAM CFD model with nonlinear FEM code for building response & SP & M & 1.0 \\ \hline
2.2.3 & Coupled OpenFOAM CFD model with nonlinear FEM code for building response & SP & M &  \\ \hline
2.3 & Quantification of Effects of Wind Bourne Debris & GC & D & \\ \hline
2.4 & Application to utilize GIS and online to account for wind speed given local terrain, topography and nearby buildings & GC & D & \\ \hline
2.5 & Ability to utilize synthetic wind loadings & SP & M & 1.0  \\ \hline
2.5.1 & per Wittig and Sinha & SP & D & 1.1  \\ \hline
3 & \textbf{Building Model Generation} & GC & 2.0 \\ \hline
3.1 & Ability to quickly create a simple nonlinear building model & GC & D & 1.1 \\ \hline
3.2  & Ability to define building and use Expert System to generate FE mesh & SP & &  \\ \hline
	3.2.1 & Expert system for Concrete Shear Walls & SP & M &  \\ \hline
	3.2.2 & Expert system for Moment Frames & SP & M &  \\ \hline
	3.2.3 & Expert system for  Braced Frames & SP & M &   \\ \hline
3.3 & Ability to define building and use Machine Learning applications to generate FE & GC &  &  \\ \hline
	3.3.1 & Machine Learning for Concrete Shear Walls & SP & M &  \\ \hline
	3.3.2 & Machine Learning for Moment Frames & SP & M &  \\ \hline
	3.3.3 & Machine Learning for Braced Frames & SP & M &   \\ \hline
	3.4 & Ability to specify connection details for member ends & SP & M & 2.2 \\ \hline
	3.5 & Ability to define a user-defined moment-rotation response representing the connection details & SP & D & 2.2 \\ \hline
	4 & \textbf{perform nonlinear Structural Analysis} &  &  \\ \hline
4.1 & Ability to use utilize existing nonlinear analysis software used in earthquake engineering & GC & M & 1.0 \\ \hline
4.1.1 & Utilize open source OpenSees software & SP & M & 1.0 \\ \hline
4.2.1 & Ability to provide own OpenSees Analysis script to OpenSees engine. & SP & D & 1.0 \\ \hline
4.3.1 & Ability to provide own Python script and use OpenSeesPy engine. & SP & O & 1.2 \\ \hline
4.2 & Ability to use alternative FEM engine & SP & M & 2.0 \\ \hline
	5 & \textbf{UQ - Method} &  GC &  \\ \hline
	5.1 & \textbf{UQ - Forward Propogation Methods} & SP  &  \\ \hline
	5.1.1 & Ability to use basic  Monte Carlo and LHS methods & SP & M & 1.0 \\ \hline
	5.1.2 & Ability to use Importance Sampling  & SP & M & 2.0 \\ \hline
	5.1.3 & Ability to use Gaussian Process Regression & SP & M & 2.0 \\ \hline
	5.1.4 & Ability to use Own External UQ Engine & SP & M &  \\ \hline
	5.2 & \textbf{UQ - Reliability Methods} & UF &  &  \\ \hline
	5.2.1 & Ability to use First Order Reliability method & UF & M &  \\ \hline
	5.2.2 & Ability to use Second Order Reliability method & UF & M & \\ \hline
	5.2.2 & Ability to use Surrogate Based Reliability & UF & M & \\ \hline
	5.2.3 & Ability to use Own External Application to generate Results & UF & M &  \\ \hline
	5.3 & \textbf{UQ - Sensitivity Methods} & UF &  &  \\ \hline
	5.3.1 & Ability to obtain Global Sensitivity Sobol's indices & UF & M &  \\ \hline
    6 & \textbf{UQ – Random Variables} &  &  \\ \hline
    6.1 & Ability to Define Variables of certain types: & GC &  &  \\ 
    6.1.1 &  Normal & SP & M  & 1.0 \\ \hline
    6.1.2 &  Lognormal & SP & M & 1.0 \\ \hline
    6.1.3 & Uniform & SP & M & 1.0  \\ \hline
    6.1.4 & Beta & SP & M & 1.0 \\ \hline
    6.1.5 & Weibull &  SP & M  & 1.0 \\ \hline
    6.1.6 & Gumbel &  SP & M & 1.0  \\ \hline
    6.2 & User defined Distribution & SP & M &  \\ \hline
    6.3 & Define Correlation Matrix & SP & M &  \\ \hline
     7 & Tool to allow user to load and save user inputs & SP & M & 1.0 \\ \hline
    8 & \textbf{Application Outputs} &  &  \\ \hline
    8.1 & Ability to see pressure distribution on building & GC & M &   \\ \hline
    8.2 & Ability to obtain basic building responses & SP & M &   \\ \hline
    8.3 & Ability to Process own Output Parameters & UF & M & 1.1  \\ \hline
    9 & \textbf{Documentation} &  &  \\ \hline
    9.1 & Documentation exists on tool usage & SP & M & 1.1  \\ \hline
    9.2 & Video Exists demonstrating usage & SP & M & 1.1  \\ \hline
    9.3 & Verification Examples Exist & SP & M & 1.1  \\ \hline
    10 & \textbf{Misc.} &  &  \\ \hline
    10.1 & Installer  which installs application and all needed software & UF & M &   \\ \hline
	\bottomrule 
\caption{Requirements for WE-UQ}
  \label{tab:featureRequirements}                 
\end{longtable}

Feature Requirements (M=Mandatory, D=Desirable, O=Optional, P=Possible Future)



\subsection{Response of Building to Earthquake Hazard}
The following are the requirements for response of single structure to earthquake hazards. The requirements are being met by the EE-UQ application. All items are related to WBS 1.3.8.

 \begin{longtable}{| p{.07\textwidth} | p{.65\textwidth} | p{.08\textwidth} | p{.08\textwidth} |  p{.08\textwidth} |}

  \caption{Requirements for EE-UQ}
  \label{tab:featureRequirements}    
     \\
   \hline
\rowcolor{lightgray}

      \# & Description & Source & Priority & Version \\ \hline
      E1 & \textbf{Ability to determine response of Building Subject to Earthquake hazard including formal treatment of randomness and uncertainty uncertainty} & GC 2.T13 & M  & 1.0  \\ \hline
 E1.1 & Ability of Practicing Engineers to use multiple coupled resources (applications, databases, viz tools) in engineering practice & GC 5.IC4 & M 1.0 & \\ \hline
E1.2 & Ability to utilize resources beyond the desktop including HPC & GC 5.IC4 & M & 1.0 \\ \hline
E1.3 & Tool should incorporate data from www & GC & M & 1.0 \\ \hline
E1.4 & Tool available for download from web & GC 2.T16 & M & 1.0 \\ \hline
E1.5 & Ability to benefit from programs that move research results into practice and obtain training & GC 2.13 & M & \\ \hhline{=====}

      E2 & \textbf{Various Motion Selection Options} & SP & M & 1.0  \\ \hline
      E2.1 & Ability to select from Multiple input motions and view UQ due to all the discrete events & GC & M & 1.0  \\ \hline
      E2.2 & Ability to select from list of SimCenter motions & SP & M & 1.0 \\ \hline
      E2.3 & Ability to select from list of PEER motions & SP & D & 1.0 \\ \hline
      E2.4 & Ability to use OpenSHA and selection methods to generate motions & UF & D & 1.0 \\ \hline
      E2.5 & Ability to Utilize Own Application in Workflow & SP & M & 1.0 \\ \hline
      E2.6 & Ability to use Broadband & SP & D &  \\ \hline
      E2.7  & Ability to include Soil Structure Interaction Effects & GC & M & 1.1 \\  \hline
      E2.7.1  & 1D nonlinear site response with effective stress analysis & SP & M & 1.1  \\ \hline
      E2.7.2  & Nonlinear site response with bidirectional loading & SP & M & 1.2 \\  \hline
      E2.7.3  & Nonlinear site response with full stochastic characterization of soil layers & SP & M &  \\ \hline
      E2.7.4 & Nonlinear site response, bidirectional different input motions  & SP & M &  \\  \hline
      E2.7.5 & Ability to couple models from point of rupture through rock and soil into structure, which represents future of professional design practice & GC GC 2.T12 & M &  \\  \hline
      E2.7.5.1 & Interface using DRM method  & SP  & M &  \\  \hline
      E2.8 & Utilize PEER NGA www ground motion selection tool  & UF & D & 2.0 \\ \hline
      E2.9 & Ability to select from synthetic ground motions & SP & M & 1.0  \\
      E2.9.1 & per Vlachos, Papakonstantinou, Deodatis (2017) & SP & D & 1.1  \\ 
      E2.9.2 & per Dabaghi, Der Kiureghian (2017) & UF & D & 2.0 \\  \hhline{=====}
      
      
	E3 & \textbf{Building Model Generation} & GC & M & 1.0 \\ \hline
	E3.1 & Ability to quickly create a simple nonlinear building model for simple methods of seismic evaluation & GC 2.T13 & D & 1.1 \\ \hline
	E3.2 & Ability to use existing OpenSees model scripts & SP & M & 1.0 \\ \hline
	E3.3  & Ability to define building and use Expert System to generate FE mesh & SP & &  \\ \hline
	E3.3.1 & Expert system for Concrete Shear Walls & SP & M &  \\ \hline
	E3.3.2 & Expert system for Moment Frames & SP & M &  \\ \hline
	E3.3.3 & Expert system for  Braced Frames & SP & M &   \\ \hline
	E3.4 & Ability to define building and use Machine Learning applications to generate FE & GC &  &  \\ \hline
	E3.4.1 & Machine Learning for Concrete Shear Walls & SP & M &  \\ \hline
	E3.4.2 & Machine Learning for Moment Frames & SP & M &  \\ \hline
	E3.4.3 & Machine Learning for Braced Frames & SP & M &   \\ \hline
	E3.5 & Ability to specify connection details for member ends & UF & M & 2.2 \\ \hline
	E3.6 & Ability to define a user-defined moment-rotation response representing the connection details & UF & D & 2.2 \\  \hhline{=====}
	
	
	E4 & \textbf{Perform Nonlinear Analysis} & GC & M & 1.0 \\ \hline
	E4.1 & Ability to specify OpenSees as FEM engine and to specify different analysis options & SP & M & 1.0 \\ \hline
	E4.2 & Ability to provide own OpenSees Analysis script to OpenSees engine. & SP & D & 1.0 \\ \hline
	E4.3 & Ability to provide own Python script and use OpenSeesPy engine. & SP & O & 1.2 \\ \hline
	E4.4 & Ability to use alternative FEM engine. & SP & M & 2.0 \\ \hhline{=====}

UQ & \textbf{Ability to use various UQ Methods} & GC & M &  \\ \hline
UQ1 & \textbf{Forward Propagation Methods} & GC  & M & 1.0 \\ \hline
UQ1.1 & Ability to use basic Monte Carlo and LHS methods & SP & M & 1.0 \\ \hline
UQ1.2 & Ability to use Importance Sampling  & SP & M & 2.0 \\ \hline
UQ1.3 & Ability to use Gaussian Process Regression & SP & M & 2.0 \\ \hline
UQ1.4 & Ability to use Own External UQ Engine & SP & M &  \\ \hline
UQ2 & \textbf{Ability to use various Reliability Methods} & UF & M & 1.0 \\ \hline
UQ2.1 & Ability to use First Order Reliability method & UF & M &  \\ \hline
UQ2.2 & Ability to use Second Order Reliability method & UF & M & \\ \hline
UQ2.2 & Ability to use Surrogate Based Reliability & UF & M & \\ \hline
UQ2.3 & Ability to use Own External Application to generate Results & UF & M &  \\ \hline
UQ3 & \textbf{Ability to user various Sensitivity Methods} & UF & M & 1.0  \\ \hline
UQ3.1 & Ability to obtain Global Sensitivity Sobol's indices & UF & M &  \\ \hline
UQ4 & \textbf{Various Random Variable Probability Distributions} & SP & M & 1.0 \\ \hline
UQ4.1 & Ability to Define Variables of different types: & SP & M & 1.0  \\ \hline
UQ4.1.1 & Normal & SP & M  & 1.0 \\ \hline
UQ4.1.2 & Lognormal & SP & M & 1.0 \\ \hline
UQ4.1.3 & Uniform & SP & M & 1.0  \\ \hline
UQ4.1.4 & Beta & SP & M & 1.0 \\ \hline
UQ4.1.5 & Weibull &  SP & M  & 1.0 \\ \hline
UQ4.1.6 & Gumbel &  SP & M & 1.0  \\ \hline
UQ4.2 & User defined Distribution & SP & M &  \\ \hline
UQ4.3 & Define Correlation Matrix & SP & M &  \\ \hline
UQ4.4 & Random Fields & SP & M &  \\ \hline
>>>>>>> b55c16b9402fc3444addf52182320452de7e27c2




    EE & \textbf{Education} &  &  & \\ \hline
    EE1 & Ability to use educational provisions to gain interdisclipinary education so as to gain expertise in earth sciences and physics, engineering mechanics, geotechnical engineering, and structural engineering in order to be qualified to perform these simulations & GC 2.T12 & D & \\ \hline
    EE2 & Documentation exists on tool usage & SP & M & 1.1  \\ \hline
    EE3 & Video Exists demonstrating usage & SP & M & 1.1  \\ \hline
    EE4 & Verification Examples Exist & SP & M & 1.1  \\ \hhline{=====}
    EM & \textbf{Misc.} &  &  \\ \hline
    EM1 & Tool to allow user to load and save user inputs & SP & M & 1.0 \\ \hline
    EM2 & \textbf{Engineering Demand Parameters} &  &  \\ \hline
    EM3 & Ability to Process own Output Parameters & UF & M & 1.1  \\ \hline
    EM4.2 & Add to Standard Earthquake a variable indicating analysis failure & UF & D &   \\ \hline
    EM5 & Add to Standard Earthquake a variable indicating analysis failure & UF & D &   \\ \hline
    EM6 & Installer which installs application and all needed software & UF & M &   \\ \hline
      
  \bottomrule 
               
\end{longtable}

\noindent
KEY:\\
Source: GC=Needed for Grand Challenges, SP=Senior Personnel, UF=User Feedback \\
Need: M=Mandatory, D=Desirable, P=Possible Future \\
Version: Version number the basic requirement was met 



\subsection{Performance Based Engineering}
The following are the requirements for response of single structure to earthquake hazards. The requirements are being met by the EE-UQ application.


%\nocite{*}

% \appendix
% \chapter{More Monticello Candidates}

\pagestyle{plain}
{
  \renewcommand{\thispagestyle}[1]{}	
  \printbibliography           
}

\end{document}
